\documentclass[a4paper,12pt]{amsart}

\usepackage{globaldef}
% DEFINITIONS PDFLATEX

%%%%%%%%%%%%%%%%%%%%%%%%
% FONTS AND LANGUAGES
%%%%%%%%%%%%%%%%%%%%%%%% 
\usepackage[T1]{fontenc} % In case I am compiling with Latex. UTF-8 is not necessary anymore
\usepackage[utf8]{inputenc}

%%%%%%%%%%%%%%%%%%%%%%%%
% ADDITIONAL PACKAGES
%%%%%%%%%%%%%%%%%%%%%%%% 
\usepackage{parskip}
\usepackage{tikz} % Drawing package
\usepackage[backend=biber, style=alphabetic, sorting=ydnt]{biblatex}
%\usepackage[backend=biber, style=alphabetic]{biblatex}
\usepackage{listings}
\usepackage{ytableau}


\usepackage{mdframed} % Better than the framed package


\usetikzlibrary{intersections,decorations.text} % this is to make the cover
\usetikzlibrary{matrix, arrows.meta}            % this is to improve diagrams

\setlength\parindent{20pt}

%%%%%%%%%%%%%%%%%%%%%%%%
% METADATA
%%%%%%%%%%%%%%%%%%%%%%%% 

% See options in the official manual
\hypersetup{ 
	pdftitle={integrable systems, programming and so on},
	pdfsubject={High Energy Physics}, 
	pdfauthor={thiago araujo},
	pdfkeywords={gauge; susy; strings; fields; cft; python},
	colorlinks=true,    % false: color frames ; true: color links
    linkcolor=myPurple, % Color of internal links (sections, pages and so on)
    citecolor=myPurple, % color for bibliographical citations
    urlcolor =myPurple, % color for linked URL
    linktocpage=true    % link page to table of contents
}

% Options for listings package I saw the sniipet below here: 
% https://stackoverflow.com/questions/3175105/inserting-code-in-this-latex-document-with-indentation
\lstset{frame=tb,
  language=Python,
  aboveskip=3mm,
  belowskip=3mm,
  showstringspaces=false,
  columns=flexible,
  basicstyle={\small\ttfamily},
  numbers=none,
  numberstyle=\tiny\color{myPurple},
  keywordstyle=\color{myRed},
  commentstyle=\color{myBlue},
  stringstyle=\color{myPurple!80},
  breaklines=true,
  breakatwhitespace=true,
  tabsize=3
}

%%%%%%%%%%%%%%%%%%%%%%%% 
% COLORS
%%%%%%%%%%%%%%%%%%%%%%%% 
% see palette here: https://github.com/enkia/tokyo-night-vscode-theme
\definecolor{myPurple}{RGB}{90, 74, 120}
\definecolor{myBlue}{RGB}{15, 75, 110}
\definecolor{myRed}{RGB}{191,97,106}
\definecolor{myDarkGray}{RGB}{216, 222, 233}
\definecolor{myLightGray}{RGB}{236, 239, 244}

\definecolor{c1}{RGB}{129, 162, 193} % myBlue {15, 75, 110}
\definecolor{c2}{RGB}{216, 222, 233} % myDarkGray
\definecolor{c3}{RGB}{236, 239, 244} % myLightGray
\definecolor{c4}{RGB}{59, 66, 82}
\definecolor{c5}{RGB}{76, 86, 106}

%%%%%%%%%%%%%%%%%%%%%%%%
% THEOREM
%%%%%%%%%%%%%%%%%%%%%%%% 
\newtheorem{theorem}{Theorem}
\newtheorem{corollary}{Corollary}
\newtheorem{proposition}{Proposition}
\newtheorem{conjecture}{Conjecture}
\newtheorem{lemma}{Lemma}
\newtheorem{example}{Example}
\newtheorem{exercise}{Exercise}
\newtheorem{notation}{Notation}
\newtheorem{remark}{Remark}
\newtheorem{definition}{definition}

%%%%%%%%%%%%%%%%%%%%%%%%
% MATH OPERATORS
%%%%%%%%%%%%%%%%%%%%%%%% 
\DeclareMathOperator{\Res}{Res}
\DeclareMathOperator{\Tr}{Tr}

%%%%%%%%%%%%%%%%%%%%%%%%
% MACROS
%%%%%%%%%%%%%%%%%%%%%%%% 
\DeclarePairedDelimiter{\bra}{\langle}{\rvert}
\DeclarePairedDelimiter{\ket}{\lvert}{\rangle}
\DeclarePairedDelimiter{\bbra}{\langle\!\langle}{\rvert}
\DeclarePairedDelimiter{\kket}{\lvert}{\rangle\!\rangle}
\DeclarePairedDelimiterX{\bracket}[2]{\langle}{\rangle}{#1\vert#2}
\DeclarePairedDelimiterX{\bbracket}[2]{\langle\!\langle}{\rangle\!\rangle}{#1\vert#2}


\bibliography{/home/thiago/Sync/wiki/database/bib-database.bib}

%% Extra packages 
\usepackage{amsaddr}                % Affiliation in the first page
\usepackage{slashed}                % Feynman slash notation
\usepackage{simplewick}             % Wick contractions
\usepackage{fontawesome}
\usepackage{tikz-cd}
\usepackage[textwidth=20mm]{todonotes}

\begin{document}

%%%%%%%%%%%%%%%%%%%%%%%%%%%%%%%%%%%%%%%%%%%%%%%%%%
%%%%%%%%%%%%%%%%%%%%%%%%%%%%%%%%%%%%%%%%%%%%%%%%%%

\title{Slavnov products and KP tau functions}

\author{Thiago Araujo}


\address{\noindent 
Instituto de Ciências Exatas, Departamento de Física\\
Universidade Federal Fluminense\\
Rua Des. Ellis Hermydio Figueira, 783, Aterrado\\
27213-145 Volta Redonda, RJ, Brazil
}

\email{\texttt{\href{thiaraujo@id.uff.br}{thiaraujo@id.uff.br}}}

\begin{abstract}
One important ingredient in the Algebraic Bethe Ansatz is the scalar
product of Bethe states.  There are many important results on this
topic and references therein, but here we turn our attention to the
well known fact that scalars products in the algebraic Bethe ansatz
can be written as determinants.  More specifically, we will
investigate a particular type of correlation functions in quantum
integrable spin chains, the so called \emph{Slavnov product} These
objects can be seen as building blocks of more sofisticated
correlation functions and are instrumental in the study of norms of
Bethe wave functions But more importantly for our present discussion,
Slavnov products seem to be a bridge connecting quantum and classical
integrable system.
\end{abstract}

\date{\today}
\keywords{Integrability, Bethe states, Schur, Hall-Littlewood, Toda, KP}
\subjclass[2020]{82B20, 82B23}

\maketitle

\setcounter{tocdepth}{2}
\tableofcontents


%%%%%%%%%%%%%%%%%%%%%%%%%%%%%%%%%%%%%%%%%%%%%%%%%%
\section{Introduction}
%%%%%%%%%%%%%%%%%%%%%%%%%%%%%%%%%%%%%%%%%%%%%%%%%%

One important ingredient in the Algebraic Bethe Ansatz is the scalar
product of Bethe states.  There are many important results on this
topic, see~\cite{Korepin:1993kvr} and references therein, but here we
turn our attention to the well known fact that scalars products in
the algebraic Bethe ansatz can be written as determinants.

More specifically, we will investigate a particular type of
correlation functions in quantum integrable spin chains, the so called
\emph{Slavnov product}~\cite{Slavnov1989}. These objects can be seen as building blocks
of more sofisticated correlation functions and are instrumental in the
study of norms of Bethe wave functions~\cite{Korepin:1993kvr}.  But
more importantly for our present discussion, Slavnov products seem to
be a bridge connecting quantum and classical integrable system.

The study of connections between classical and quantum integrable
systems is an old subject, and has been addressed from a wide range of
perspectives~\cite{Wu:1975mw, Its:1992bj, korepin2000the, Foda2009,
  Alexandrov:2011aa}.  In~\cite{Foda:2009zz}, the authors started the
analysis of the connections between the Slavnov products of the
Heisenberg XXZ spin chain, and the Kadomtsev–Petviashvili (KP)
classical integrable hierarchy. This analysis has been extended to
some directions and applications, for example~\cite{Wheeler:2010vmq,
  Foda:2010, Takasaki:2010qm, Foda:2012wn, Foda:2012wf}.
In~\cite{Araujo:2021ghu, Araujo:2024klz}, the author of the current
paper also investigated some developments of this research field. In
particular, the relation between integrable hierarchies and quantum
integrable systems such as the Q-Boson integrable system and in the
Temperley-Lieb open spin chains.

Although the specific details of he Slavnov product are radically
different, In all these cases discussed in the previous paragraph, the
general functional structure of the Slavnov products of different
models are basically the same. More specifically, module some
multiplicative terms, all these expressions are determinant, and it is
the fact that allows us to manipulate these objects to show that thesy
are KP and Toda tau functions.

Belliard and Slavnov~\cite{Belliard:2019bfz} have addressed precisely
this property: \emph{Why scalar products in the algebraic Bethe ansatz
have determinant representations}.  Using their results, we now want
to understand \emph{why and how (some) scalar products in the
algebraic Bethe ansatz are KP and Toda tau functions}. In the current
show that as long as the integrable system satisfies the conditions
established in~\cite{Belliard:2019bfz} -- and some additional
restrictions -- the Slavnov product is guaranteed to satisfy the KP
integrable hiearchy. Moreover, we also discuss some immediate
consequences of this result.

In section 2 we review the main aspects of the
work~\cite{Belliard:2019bfz} and fix the main notation we use in the
remainder of this work. (...)






%%%%%%%%%%%%%%%%%%%%%%%%%%%%%%%%%%%%%%%%%%%%%%%%%%
\section{Scalar products as determinants}
%%%%%%%%%%%%%%%%%%%%%%%%%%%%%%%%%%%%%%%%%%%%%%%%%%

In order to make the paper as self-contained as possible, this section
reviews the main arguments in~\cite{Belliard:2019bfz}. Let us start
with a set of arbitrary complex numbers \(\bm{u} =
\{u_j\}_{j=1}^{n+1}\), and we set \(n+1\) sets
\(\bm{u}_j=\bm{u}\setminus \{u_j\}\). In the algebraic Bethe ansatz
setting~\cite{Korepin:1993kvr}, we can now define \(n+1\) off-shell
Bethe vectors \(\ket{\Psi(\bm{u}_j)}\), i.e. Bethe states where the
algebraic Bethe equations are not imposed on
\(\bm{u}_j\). Additionally, we also consider a set of Bethe roots
\(\bm{v} = \{ v_k \}_{k=1}^n\), and the on-shell Bethe vectors
\(\ket{\Psi(\bm{v})}\).


\subsection{Determinant representation}
The action of the transfer matrix \(\mathcal{T}(z)\), that is an
Hermitian operator, on the dual Bethe vector is given by
\begin{equation}
\label{eq:eigenvalue-bethe}
\bra{\Psi(\bm{v})}\mathcal{T}(z) = \Lambda(z; \bm{v}) \bra{\Psi(\bm{v})} \; , 
\end{equation}
where \(\Lambda(z; \bm{v}) \) is the eigenvalue of the transfer matrix. 

We now introduce \(n+1\) scalar products between the on-shell and
off-shell Bethe states
\begin{equation}
  \bm{\zeta}_j(\bm{u}_j, \bm{v}) \equiv  \bm{\zeta}_j
  = \bracket{\Psi(\bm{v})}{\Psi(\bm{u}_j)}\qquad j =1, \dots, n+1\; .
\end{equation}
These partially on-shell scalar products are called \emph{Slavnov
products}.

Before procceding, let us also establish the class of models we are
dealing with.  Let us consider a class of integrable models where the
action of the transfer matrix on a generic off-shell bethe states can
be expanded as
\begin{subequations}
\begin{equation}
\label{eq:scalar-exp}
  \mathcal{T}(u_j)\ket{\Psi(\bm{u}_j)}  = \sum_{k=1}^{n+1} L_{jk} \ket{\Psi(\bm{u}_k)} \; ,
\end{equation}
where \(L_{jk}\) are numerical coefficients. Observe that the terms
\(L_{jk}\) with \(j\neq k\) contain the unwanted terms.; and these
non-diagonal terms must vanish when the Bethe equations are
satisfied. Therefore, since the diagonal coefficient survives, that is
\(L_{jj}\), we conclude that these terms must be equal to the
eigenvalue of the tranfer matrix, that is
\begin{equation}
  \mathcal{T}(u_j)\ket{\Psi(\bm{u}_j)} =  \Lambda(u_j; \bm{u}_j) \ket{\Psi(\bm{v})} + 
  \sum_{\substack{k=1\\ k\neq j}}^{n+1} L_{jk} \ket{\Psi(\bm{u}_k)} \; . 
\end{equation}
\end{subequations}
This class of models include a wide range of known integrable systems,
including the ones discussed in the introduction. One important class
of models that does not fall in this classification are with elliptic
R-matrices, although some of these models have slavnov products with
determinantal representations.

\begin{proposition}
The Slavnov products \(\bm{\zeta}_j\) satisfy a system of linear
equations.
\end{proposition}

\paragraph{\textbf{\emph{Proof}}} Let us first use that
\begin{equation}
\label{eq:scalar-prod}
  \bra{\Psi(\bm{v})} \left(\mathcal{T}(u_j)\ket{\Psi(\bm{u}_j)} \right) =
  \left(\bra{\Psi(\bm{v})}\mathcal{T}(u_j)\right)\ket{\Psi(\bm{u}_j)}
  \qquad j =1, \dots, n+1\; .
\end{equation}
The right-hand side of this equation can be simplified with the eigenvalue
expression~(\ref{eq:eigenvalue-bethe}), that is
\begin{equation}
  \left(\bra{\Psi(\bm{v})}\mathcal{T}(u_j)\right)\ket{\Psi(\bm{u}_j)}  =
\Lambda(u_j; \bm{v}) \bracket{\Psi(\bm{v})}{\Psi(\bm{u}_j)} = 
\Lambda(u_j; \bm{v}) \bm{\zeta}_j\; . 
\end{equation}

Let us now use the expansion~(\ref{eq:scalar-exp}) on the left-hand
side of~(\ref{eq:scalar-prod}). Therefore, we have
\begin{equation}
  \bra{\Psi(\bm{v})} \left(\mathcal{T}(u_j)\ket{\Psi(\bm{u}_j)} \right) =
  \sum_{k=1}^{n+1} L_{jk} \bm{\zeta}_k \; .
\end{equation}

Puttting all these facts together, we get
\begin{equation}
  \sum_{k=1}^{n+1} L_{jk} \bm{\zeta}_k  = 
  \Lambda(u_j; \bar{v}) \bm{\zeta}_j\quad \Rightarrow \quad
  \sum_{k=1}^{n+1} M_{jk} \bm{\zeta}_k = 0 \; ,
\end{equation}
where
\begin{equation}
  M_{jk} = L_{jk} - \delta_{jk}\Lambda(u_j; \bm{v})\; . 
\end{equation}
And this expression shows that the Slavnov products satisfy a linear system.
We can write this system in a matrix form as 
\begin{equation}
\bm{M} \bm{\zeta} = 
  \left(\bm{L} - \bm{\Lambda} \right) \bm{\zeta} = \bm{0}\; ,
\end{equation}
where
\begin{subequations}
  \begin{equation}
    \bm{L} = 
 \begin{pmatrix}
   L_{1,1} & \cdots & L_{1,n+1}\\
   L_{2,1} & \cdots & L_{2,n+1}\\
   \vdots & \vdots & \vdots\\
   L_{n+1, 1} & \cdots & L_{n+1, n+1}\\
 \end{pmatrix}  \quad , \quad
    \bm{\Lambda} = \mathrm{diag}(\Lambda_1, \Lambda_2, \dots, \Lambda_{n+1})
\end{equation}
with \(\Lambda_j \equiv \Lambda(u_j; \bm{v})\), and also
\begin{equation}
  \bm{\zeta} = (\bm{\zeta}_1, \dots , \bm{\zeta}_{n+1})^T\qquad \, 
  \bm{0} = (0, \dots , 0)^T\; .
\end{equation}
\end{subequations}
 \qed

There are two important consequences for us now. First of all, if we
assume that this system has a non-trivial solution, it will be written
as minors of the matrix \(\bm{M}\).

Moreover, since the system is homogeneous, the solutions are
determined up to multiplicative terms, that we can fix them by
requiring that the final result gives the required tau function. In
the original works of Foda and collaborators, e.g.~\cite{Foda:2009zz,
  Wheeler:2010vmq}, the authors perform a series of redefinitions to
reach the same goal.

Evidently being a determinant is not enough to guarantee that the
Slavnov product is a tau function.  In order to accomplish this
result, we need to specify the models. Following the original
work~\cite{Belliard:2019bfz}, we will only discuss the rational models
-- we will see that the calculations are overwhelming even in this
simple case. The generalization for open boundary conditions and for
trigonometric models is immediate, see~\cite{Slavnov:2019aba} for more
details on the generalization.

%%%%%%%%%%%%%%%%%%%%%%%%%%%%%%%%%%%%%%%%%%%%%%%%%%
%%%%%%%%%%%%%%%%%%%%%%%%%%%%%%%%%%%%%%%%%%%%%%%%%%

\subsection{Rational models}

Let us startwith an R-matrix of the form
\begin{equation}
  R(u,v) = \mathbb{1} + g(u,v) \mathcal{P}\; , \qquad
  g(u,v) = \frac{c}{u - v}
\end{equation}
where \(c\) is a constant, the identity operator is denoted by
\(\mathbb{1}\) and \(\mathcal{P}\) is the permutation operator.

In the models with rational R-matrix, the eigenvalues of the transfer matrix
have the form 
\begin{equation}
  \Lambda(z, \bm{v}) = g(z, \bm{v}) \mathcal{Y}(z; \bm{v})\; ,
\end{equation}
where the function \(\mathcal{Y}(z; \bm{v})\) is symmetric over the
Bethe roots \(\bm{v}\), and linearly depends on every
\(v_j\).

Generically, it is written as
\begin{equation}
  \mathcal{Y}(z; \bm{v}) = \sum_{p=0}^n \alpha_p(z) \sigma_p^{(n)}(\bm{v})\; , 
\end{equation}
where \(\sigma_p^{(n)}(\bm{v})\) are elementary symmetric polynomials
in \(\bm{v}\) and where \(\alpha_p(z)\) are free functional
parameters. In fact, in the XXX discussed in~\cite{Belliard:2019bfz},
these functions all \(\alpha_p(z)\) are polynomials of degree \(2n\).
The important point for us is that the functions \(\alpha_p(z)\) are
not singular when we consider the limit \(z \to v_j \in \bm{v}\), and
that also means that \(\mathcal{Y}(z; \bm{v})\) is also regular in
this limit.

Let us also define the following product 
\begin{equation}
  g(z, \bm{v}) = \prod_{v_i\in \bm{v}} g(z; v_i)\; , 
\end{equation}
and that means that each Bethe root \( v_i \in \bm{v}\) 
is a pole of \(g(z;\bm{v})\). Therefore, if \(\Gamma\) is a contour
containing all the Bethe roots \(\{v_j\}_{j=1}^n\), we have 
\begin{equation}
  \frac{1}{2 \pi i}\oint_{\Gamma} \mathrm{d}z g(z; \bm{v}) = \sum_{j=1}^n  \prod_{v_i \in \bm{v}\setminus \{u_j\}} g(z, v_i)\; .
\end{equation}
Furthermore, the coefficients \(L_{jk}\) are given by
\begin{equation}
  L_{jk} = g(u_k, \bm{u}_k) \mathcal{Y}(u_k; \bm{u}_j)
\end{equation}
and from this expression one can calculate the matrix \(\bm{M}\).

It has also been shown~\cite{Belliard:2019bfz} that if \(\det
(\bm{M}) = 0 \) and \(\mathrm{rank}(\bm{M}) = n\), then, the Slavnov
products are given by
\begin{equation}
 \bm{\zeta}_\ell = \phi(\bm{v}) \tilde{\Delta}(\bm{u}_\ell)\hat{\Omega}_\ell\; ,
\end{equation}
where \(\phi(\bm{v})\) is a function of the Bethe roots and
\begin{equation}
  \tilde{\Delta}(\bm{u}_\ell) = \prod_{\substack{u_j, u_j \in \bm{u}_\ell \\ j>k}} \frac{c}{u_j - u_k}\quad \Rightarrow \quad
\frac{\tilde{\Delta}(\bm{u}_\ell)}{c_0} = \prod_{\substack{u_j, u_j \in \bm{u}_\ell \\ j>k}} \frac{1}{u_j - u_k}\; , 
\end{equation}
where we have absorbed the product of \(c\) into a new constant \(c_0\). 
All in all, \(\tilde{\Delta}(\bm{u}_\ell)/c_0\) is the inverse of the Vandermonde determinant
\(\Delta(\bm{u}_\ell) = \prod_{j>k}(u_j - u_k)\).

Additionally, the matrices \(\hat{\Omega}_\ell\) are minors of
\(\bm{\Omega}\) with the \(\ell\)-th column excluded, in other words,
\begin{equation}
 \hat{\Omega}_\ell = \det_{k\neq \ell} \Omega_{jk}\; ,
\end{equation}
where
\begin{equation}
  \Omega_{jk}(u_k; \bm{v}) = g(u_k, v_j) \mathcal{Y}(u_k; \{u_k, \bm{v}_j\})\; ,
  \quad j= 1, \dots, n\; , \quad k =1, \dots, n+1\; .
\end{equation}

All in all, we have an expression of the form
\begin{equation}
  \label{eq:slavnov-tau}
  \bm{\tau}_\ell(\bm{u}_\ell; \bm{v}) \equiv \frac{1}{c_0}\frac{\bm{\zeta}_\ell}{\phi(\bm{v})}
  = \frac{\hat{\Omega}_{\ell}}{\Delta(\bm{u}_\ell)} \; .
\end{equation}
This is the most important result for us, and our goal now is to show
that it is a tau-function of the KP-hierarchy.

%%%%%%%%%%%%%%%%%%%%%%%%%%%%%%%%%%%%%%%%%%%%%%%%%%
\section{Tau functions}
%%%%%%%%%%%%%%%%%%%%%%%%%%%%%%%%%%%%%%%%%%%%%%%%%%

In order to proceed, let us simplify the notation and organize the
results we described before. First of all, we need to observe
that we can define \(n\) functions \(\Omega_j(z)\) of the form
\begin{equation}
\begin{split}
  \Omega_{j}(z; \bm{v}) & = g(z, v_j) \mathcal{Y}(z; \{z, \bm{v}_j\})\qquad j =1, \dots, n\\
   & = \frac{c}{z - v_j} \mathcal{Y}(z; \{z, \bm{v}_j\})\; .
\end{split}
\end{equation}
Moreover, we have the expansion 
\begin{equation}
  \mathcal{Y}(z; \{z,\bm{v}_j\}) = \sum_{p=0}^n \alpha_p(z) \sigma_p^{(n)}(\{z,\bm{v}_j\})\; , 
\end{equation}
and using that the elementary symmetric polynomials decompose as
\begin{equation}
  \sigma_p^{(n)}(\{z,\bm{v}_j\}) = \sigma_p^{(n-1)}(\bm{v}_j) + z \sigma_{p-1}^{(n-1)}(\bm{v}_j) \; . 
\end{equation}

Let us also denote the functions
\begin{equation}
  \label{eq:y-functions-alpha}
  \mathcal{Y}_j(z)  \equiv  \mathcal{Y}(z; \{z,\bm{v}_j\})  = 
\sum_{p=0}^{n} \alpha_p(z)
  \left(\sigma_p^{(n-1)}(\bm{v}_j) + z \sigma_{p-1}^{(n-1)}(\bm{v}_j) \right) \; .
\end{equation}
Observe that the defining property of these functions
\(\mathcal{Y}_j(z)\) is the absence of the \(j\)-th Bethe root
\(v_j\).
Using that the elementary symmetric polynomials also satisfy \(\sigma_n^{(n-1)}(\bm{v}_j)=0\)
and \(\sigma_{-1}^{(n)}(\bm{v}_j)= 0\), we can write
\begin{equation}
  \label{eq:y-functions}
  \mathcal{Y}_j(z)  = 
  \sum_{p=0}^{n-1} 
   \left(\alpha_p(z) + z \alpha_{p+1}(z) \right) \sigma_p^{(n-1)}(\bm{v}_j) \; .
\end{equation}

Putting all these facts together, we have the functions
\begin{equation}
  \Omega_{j}(z; \bm{v}) = \frac{c}{z - v_j} \mathcal{Y}_j(z) \; .
\end{equation}
It is also convenient to consider \(c=1\), consequently \(c_0=1\).



\subsection{Alternant expression for the Slavnov products}

The matrix \(\bm{\Omega}\) can be written as
\begin{equation}
  \bm{\Omega} =
  \begin{pmatrix}
    \Omega_1(u_1) & \dots & \Omega_1(u_n) & \Omega_1(u_{n+1}) \\
    \Omega_2(u_1) & \dots & \Omega_2(u_n) & \Omega_2(u_{n+1}) \\
    \vdots & \vdots & \vdots & \vdots \\
    \Omega_n(u_1) & \dots & \Omega_n(u_n) & \Omega_n(u_{n+1})\; .
  \end{pmatrix}
\end{equation}
From this expression, we can also define a set of \(n\) square matrices
\(\bm\Omega^{(\ell)}(\bm{u}_\ell; \bm{v})\),
\(\ell = 1, \dots, n+1\), removing the \(\ell\)-th column of
\(\bm{\Omega}\), that is
\begin{equation}
  \bm{\Omega}^{(\ell)} =
  \begin{pmatrix}
    \Omega_1(u_1) & \dots & \Omega_1(u_{\ell-1}) & \Omega_1(u_{\ell+1})&\dots & \Omega_1(u_{n+1}) \\
    \vdots & \dots & \vdots & \vdots & \dots & \vdots \\
    \Omega_n(u_1) & \dots & \Omega_n(u_{\ell-1}) & \Omega_n(u_{\ell+1})&\dots & \Omega_n(u_{n+1})
  \end{pmatrix}\; .
\end{equation}

Finaly, we can write~(\ref{eq:slavnov-tau}) as 
\begin{equation}
\label{eq:slavnov-tau-functions}
  \bm{\tau}_\ell(\bm{u}_\ell; \bm{v}) = \frac{\det [\Omega_j((\bm{u}_\ell)_k)]|_{j,k=1}^n }{\Delta(\bm{u}_\ell)} 
\end{equation}
where \((\bm{u}_\ell)_k \) is the \(k\)-th component of the vector
\(\bm{u}_\ell\).  Therefore, for a fixed \(\ell\), we define the
vector \( \bm{z} \equiv \bm{u}_\ell\).  Consequently,
\begin{equation}
\begin{split}
\label{eq:slav-tau-function}
  \bm{\tau}(\bm{z}, \bm{v})
  & = \frac{1}{\Delta(\bm{z})} \det[\Omega_j(z_k) ]_{j,k=1}^n \\
  & = \frac{1}{\Delta(\bm{z})} \det\left[\frac{\mathcal{Y}_j(z_k)}{z_k - v_j} \right]_{j,k=1}^n  \; .
\end{split}
\end{equation}
All these tau functions have the form of an \emph{alternant
determinant} devided by the Vandermonde determinant.

It is well documented that the functions defined as
in~(\ref{eq:slav-tau-function}) are tau functions of the
KP-hierachy, see~\cite{Araujo:2021ghu} and references therein, in
particular~\cite{Kharchev:1991cy}. Consequently, these functions are
tau functions. 

Henceforth, we will only refer to the matrix
\(\bm{\Omega}\) defined by the components
\begin{equation}
  \label{eq:omega-matrix}
  \Omega_{jk} = \frac{\mathcal{Y}_j(z_k)}{z_k - v_j}\; .
\end{equation}



\subsection{Tau functions expansion of the Slavnov product}

We can immediately observe that the poles of the tau
function~(\ref{eq:slav-tau-function}) are exactly the Bethe roots of
the integrable model.  We now want to analyse this pole structure of
\(\Delta(\bm{z})\tau(\bm{z}, \bm{v})\).  Let us consider that the
Bethe roots are not degenerate, that is, if \(j \neq k\), then \(v_j
\neq v_k\). In that case, the poles \(v_j\) are simple poles.

Let us now consider a particular coefficient \(\bm{z}\), say \(z_l\), and
make a Laplace expansion in the column \(l\), that is
\begin{equation}
\label{eq:minor-expansion}
\begin{split}
  \bm{\tau}(\bm{z}, \bm{v})
  & = \frac{1}{\Delta(\bm{z})} \det[\bm{\Omega}(\bm{z}) ] \\
  & = \frac{1}{\Delta(\bm{z})} \sum_{j=1}^n (-1)^{j + l} \frac{\mathcal{Y}_j(z_l)}{z_l - v_j  } 
  \det[\hat{\Omega}_{j, l}]\; ,
\end{split}
\end{equation}
where \(\hat{\Omega}_{j,l} \) is the \((j, l)\) minor of
\(\bm{\Omega}\).  Now, we can extract the residue of the tau function
for \(z_l\) at the point \(v_j\), that is
\begin{equation}
  \Res_{(z_l, v_j)}(\Delta(\bm{z})\tau(\bm{z}, \bm{v})) = (-1)^{j+l} \mathcal{Y}_j(v_j) \det[\hat{\Omega}_{j, l}]
\end{equation}

\begin{subequations}
Let us now decompose the Vandermonde determinant as
\begin{equation}
  \Delta(\bm{z}) = \prod_{j>k}(z_j - z_k) = \Delta(\bm{z}_l) \prod_{r<l}(z_l - z_r) \prod_{s>l}(z_s - z_l)\; ,
\end{equation}
where \(\bm{z}_l = \bm{z}\setminus \{z_l\} = (z_1, \dots, z_{l-1},
z_{l+1}, \dots, z_n) \), and we also define the function
\begin{equation}
  \Xi_l(z_l,\bm{z}_l) = \prod_{r<l}(z_l - z_r) \prod_{r>l}(z_r - z_l)\; , 
\end{equation}
then 
\begin{equation}
  \Delta(\bm{z}) = \Xi_l(z_l, \bm{z}_l) \Delta(\bm{z}_l) \; .
\end{equation}
\end{subequations}

Inserting this decomposition into~(\ref{eq:minor-expansion}), we find
\begin{equation}
\label{eq:minor-expansion-2}
  \bm{\tau}(\bm{z}, \bm{v})
   = \frac{1}{\Xi_l(z_l, \bm{z}_l)} \sum_{j=1}^n (-1)^{j + l} \frac{\mathcal{Y}_j(z_l)}{z_l - v_j} 
\left(\frac{1}{\Delta(\bm{z}_l)} \det[\hat{\Omega}_{j, l}]\right)\; .
\end{equation}

When we define the \(n\) sets of functions
\begin{equation}
  \begin{split}
    \hat{\bm{\Omega}}_{(j)}(z) & = \left(\Omega_1(z), \dots, \Omega_{j-1}(z), \Omega_{j+1}(z), \dots, \Omega_n(z) \right)\; ,
    \quad j =1, \dots, n\; ,
  \end{split}
\end{equation}
so that we can simplify the minor
expansion~(\ref{eq:minor-expansion-2}) even further; that is
\begin{equation}
\begin{split}
  \bm{\tau}(\bm{z}, \bm{v})
  & = \frac{1}{\Xi_l(z_l, \bm{z}_l)} \sum_{j=1}^n (-1)^{j + l} \frac{\mathcal{Y}_j(z_l)}{z_l - v_j} 
    \left(\frac{1}{\Delta(\bm{z}_l)} \det[\hat{\bm{\Omega}}_{(j) r}(\bm{z}_{(l) s} ]_{r, s=1}^n\right)\\ 
  & = \frac{1}{\Xi_l(z_l, \bm{z}_l)} \sum_{j=1}^n (-1)^{j + l} \frac{\mathcal{Y}_j(z_l)}{z_l - v_j} 
    \tilde{\bm{\tau}}_j(\bm{z}_l, \bm{v})\; ,
\end{split}
\end{equation}
where \(\hat{\bm{\Omega}}_{(j)r}\) is the \(r\)-th component of
\(\hat{\bm{\Omega}}_{(j)}\) and \(\hat{z}_{(l)s}\) is the \(s\)-th
component of \(\bm{z}_{(l)}\). Moreover, in the second line we have
defined the object
\begin{equation}
  \label{eq:basis-tau}
    \tilde{\bm{\tau}}_j(\bm{z}_l, \bm{v}) = 
\frac{1}{\Delta(\bm{z}_l)} \det[\hat{\bm{\Omega}}_{(j)}(\bm{z}_{(l)} ]\; .
\end{equation}
The curious aspect of this expression is that it easy to see that each term  
\(\tilde{\bm{\tau}}_j(\bm{z}_l, \bm{v})\) is a tau function itself, and that
it acts as a basis for the Slavnov's product. 
Additionally, these basis tau functions \(\tilde{\tau}_j(\bm{z}_l,
\bm{v})\) are, by construction, completely independent of \(z_l\).

Finally, we have
\begin{equation}
  \Res_{(z_l, v_j)}(\tau(\bm{z}, \bm{v}))
  = \frac{(-1)^{j + l} \mathcal{Y}_j(v_j)}{\Xi_l(v_j, \bm{z}_l)}
  \tilde{\bm{\tau}}_j(\bm{z}_l, \bm{v})
\end{equation}

In order to avoid cluttering, let us order the parameters \(\bm{z}\) and
consider these points close to the corresponding \(\bm{v}\), that is,
we consider just the residues \(z_j \to v_j\), then
\begin{subequations}
\begin{equation}
  \Res_{(z_j, v_j)}(\tau(\bm{z}, \bm{v}))
  = \frac{\mathcal{Y}_j(v_j)}{\Xi_j(v_j, \bm{z}_j)}
  \tilde{\bm{\tau}}_j(\bm{z}_j, \bm{v})
\end{equation}
or yet 
\begin{equation}
  \tilde{\bm{\tau}}_j(\bm{z}_j, \bm{v})
  = \frac{\Xi_j(v_j, \bm{z}_j)}{\mathcal{Y}_j(v_j)}
  \Res_{(z_j, v_j)}(\tau(\bm{z}, \bm{v}))
\end{equation}
\end{subequations}
Evidently, one can repeat the same ideas and take \(z_k \in \bm{z}_j\)
close to \(v_k \in \bm{v}_j\). 

Therefore, we have the interesting result that the rresidues of the
tau function \(\tau(\bm{z}, \bm{v})\) for \(z_j\) approaching the
Bethe root \(v_j\) is proportional to a tau function.

%%%%%%%%%%%%%%%%%%%%%%%%%%%%%%%%%%%%%%%%%%%%%%%%%%
\section{Homogeneous limit}
%%%%%%%%%%%%%%%%%%%%%%%%%%%%%%%%%%%%%%%%%%%%%%%%%%

In the previous section we have discussed the limit where the
parameters \(\bm{z}\) approach the Bethe roots \(\bm{v}\).  We now
want to consider the limit where \(\bm{z}\) tend to a single variable,
say \(z_k\to z_1\), \(k=2,\dots, n\).

\subsection{Wronskian formula}

Let us first consider the case where \(z_2 \to z_1\). We consider a
series expansion about \(z_1\), and the second column of~(\ref{eq:slav-tau-function}) becomes
\begin{subequations}
\begin{equation}
  \bm{\tau}(\bm{z}, \bm{v}) = \frac{1}{\Delta(\bm{z})}
  \det
  \begin{pmatrix}
    \frac{\mathcal{Y}_1(z_1)}{z_1 - v_1}  & \frac{\mathcal{Y}_1(z_1)}{z_1 - v_1} + (z_2 - z_1) \frac{\mathcal{Y}_1^{(1)}(z_1)}{z_1 - v_1}
    + \mathcal{O}(\delta^2) &
    \frac{\mathcal{Y}_1(z_3)}{z_3 - v_1} & \dots & \frac{\mathcal{Y}_1(z_n)}{z_n - v_1}\\
    \frac{\mathcal{Y}_2(z_1)}{z_1 - v_2} & \frac{\mathcal{Y}_2(z_1)}{z_1 - v_2} + (z_2 - z_1) \frac{\mathcal{Y}_2^{(1)}(z_1)}{z_1 - v_2} 
    + \mathcal{O}(\delta^2) &
    \frac{\mathcal{Y}_2(z_3)}{z_3 - v_2} &  \dots & \frac{\mathcal{Y}_2(z_n)}{z_n - v_2}\\
    &  \vdots & & \\
    \frac{\mathcal{Y}_n(z_1)}{z_1 - v_n} & \frac{\mathcal{Y}_n(z_1)}{z_1 - v_n} + (z_2 - z_1) \frac{\mathcal{Y}_n^{(1)}(z_1)}{z_1 - v_n} 
    + \mathcal{O}(\delta^2) &
    \frac{\mathcal{Y}_n(z_3)}{z_3 - v_n} &  \dots & \frac{\mathcal{Y}_n(z_n)}{z_n - v_n}
  \end{pmatrix}
\end{equation}
where we have written \(z_2 - z_1 = \delta \to 0\), and we denote by
\(\mathcal{Y}^{(n)}(z)\) the \(n\)-th derivative of a
\(\mathcal{Y}(z)\) with respect its argument.

Now one can immediatelly see that the second column is the first
column plus terms multiplied by the factor \(\delta = z_2 -
z_1\). Using elementary operations on the column, we find
\begin{equation}
  \bm{\tau}_h(\bm{z}, \bm{v}) = \lim_{z_2 \to z_1}\frac{(z_2 - z_1)}{\Delta(\bm{z})}
  \det
  \begin{pmatrix}
    \frac{\mathcal{Y}_1(z_1)}{z_1 - v_1}  & \frac{\mathcal{Y}_1^{(1)}(z_1)}{z_1 - v_1} &
    \frac{\mathcal{Y}_1(z_3)}{z_3 - v_1} & \dots & \frac{\mathcal{Y}_1(z_n)}{z_n - v_1}\\
    \frac{\mathcal{Y}_2(z_1)}{z_1 - v_2} & \frac{\mathcal{Y}_2^{(1)}(z_1)}{z_1 - v_2} &
    \frac{\mathcal{Y}_2(z_3)}{z_3 - v_2} &  \dots & \frac{\mathcal{Y}_2(z_n)}{z_n - v_2}\\
    &  \vdots & & \\
    \frac{\mathcal{Y}_n(z_1)}{z_1 - v_n} & \frac{\mathcal{Y}_n^{(1)}(z_1)}{z_1 - v_n} &
    \frac{\mathcal{Y}_n(z_3)}{z_3 - v_n} &  \dots & \frac{\mathcal{Y}_n(z_n)}{z_n - v_n}
  \end{pmatrix}\; .
\end{equation}
We can now repeat the same ideas for \(z_3\to z_1=z_2\). We find that
the third column is the sum of the first and second columns, and
derivatives multiplied by the factor \(\delta^2\sim (z_3 - z_2)(z_3 -
z_1)\). All in all, we have
\begin{equation}
  \bm{\tau}_h(\bm{z}, \bm{v}) = \lim_{z_2, z_3 \to z_1}\frac{(z_2 - z_1)(z_3 - z_1)(z_3 - z_2)}{\Delta(\bm{z})}
  \det
  \begin{pmatrix}
    \frac{\mathcal{Y}_1(z_1)}{z_1 - v_1}  & \frac{\mathcal{Y}_1^{(1)}(z_1)}{z_1 - v_1} &
    \frac{\mathcal{Y}^{(2)}_1(z_1)}{z_1 - v_1} & \dots & \frac{\mathcal{Y}_1(z_n)}{z_n - v_1}\\
    \frac{\mathcal{Y}_2(z_1)}{z_1 - v_2} & \frac{\mathcal{Y}_2^{(1)}(z_1)}{z_1 - v_2} &
    \frac{\mathcal{Y}_2^{(2)}(z_1)}{z_1 - v_2} &  \dots & \frac{\mathcal{Y}_2(z_n)}{z_n - v_2}\\
    &  \vdots & & \\
    \frac{\mathcal{Y}_n(z_1)}{z_1 - v_n} & \frac{\mathcal{Y}_n^{(1)}(z_1)}{z_1 - v_n} &
    \frac{\mathcal{Y}_n^{(2)}(z_1)}{z_1 - v_n} &  \dots & \frac{\mathcal{Y}_n(z_n)}{z_n - v_n}
  \end{pmatrix}\; .
\end{equation}

Repeating the same idea for all columns, we get that the
multiplicative factor cancels out the Vandermonde determinant, and
that the homogeneous limit gives
\begin{equation}
  \bm{\tau}_h(z_1, \bm{v}) =
  \det
  \begin{pmatrix}
    \frac{\mathcal{Y}_1(z_1)}{z_1 - v_1}  & \frac{\mathcal{Y}_1^{(1)}(z_1)}{z_1 - v_1} &
    \frac{\mathcal{Y}^{(2)}_1(z_1)}{z_1 - v_1} & \dots & \frac{\mathcal{Y}_1^{(n-1)}(z_1)}{z_1 - v_1}\\
    \frac{\mathcal{Y}_2(z_1)}{z_1 - v_2} & \frac{\mathcal{Y}_2^{(1)}(z_1)}{z_1 - v_2} &
    \frac{\mathcal{Y}_2^{(2)}(z_1)}{z_1 - v_2} &  \dots & \frac{\mathcal{Y}_2^{(n-1)}(z_1)}{z_1 - v_2}\\
    &  \vdots & & \\
    \frac{\mathcal{Y}_n(z_1)}{z_1 - v_n} & \frac{\mathcal{Y}_n^{(1)}(z_1)}{z_1 - v_n} &
    \frac{\mathcal{Y}_n^{(2)}(z_1)}{z_1 - v_n} &  \dots & \frac{\mathcal{Y}^{(n-1)}_n(z_1)}{z_1 - v_n}
  \end{pmatrix}\; .
\end{equation}

Finally, let we write \(z_1 \equiv w\) and using elementary row operations we find
\begin{equation}
  \bm{\tau}_h(w, \bm{v}) =
  \frac{1}{\prod_{k=1}^n(w - v_k)}
  \mathcal{W}[\mathcal{Y}_1, \mathcal{Y}_2, \dots, \mathcal{Y}_n](w)
  \; .
\end{equation}
\end{subequations}
The interesting aspect of this expression is that the poles of this
function are exactly the Bethe roots, and they are easily built from
the Algebraic Bethe ansatz. Additionally, the functions
\(\{\mathcal{Y}_j(w)\}_{j=1}^n\) are linearly independent if and only
if the Bethe roots are non-degenerate.


%%%%%%%%%%%%%%%%%%%%%%%%%%%%%%%%%%%%%%%%%%%%%%%%%%
\section{Examples}
%%%%%%%%%%%%%%%%%%%%%%%%%%%%%%%%%%%%%%%%%%%%%%%%%%

Let us now consider the cases \(n=2, 3\) (the case \(n=1\) is trivial)
to gain some understanding of the problem.

\subsection{n=2} In this case, we have \(\mathcal{Y}_j\) for \(j=1,2\) and two Bethe roots
\(\bm{v} = (v_1, v_2)\). Moreover, we have the two sets \(\bm{v}_1 =
\{v_2\}\) and \(\bm{v}_2 = \{v_1\}\). From (\ref{eq:y-functions}), we have
\begin{subequations}
\begin{equation}
  \begin{split}
    \mathcal{Y}_j(z) =
    \alpha_0(z) \sigma_0^{(1)}(\bm{v}_j) +  
    \alpha_1(z)( \sigma_1^{(1)}(\bm{v}_j) + z \sigma_0^{(1)}(\bm{v}_j)) +
    \alpha_2(z) z \sigma_1^{(1)}(\bm{v}_j) \; , 
  \end{split}
\end{equation}
finally, the explicit formulas for the elementary symmetric polynomials are \(\sigma_0^{(1)}(x) =1\)
and \(\sigma_1^{(1)}(x) = x\). 

Therefore
\begin{equation}
  \begin{split}
    \mathcal{Y}_1(z) & = \alpha_0(z) + z \alpha_1(z)  + v_2 \left( \alpha_1(z) + z \alpha_2(z) \right) \\ 
    \mathcal{Y}_2(z) & = \alpha_0(z) + z \alpha_1(z)  + v_1 \left( \alpha_1(z) + z \alpha_2(z) \right) ; .
  \end{split}
\end{equation}
Consequently, we find
\begin{equation}
\begin{split}
  \bm{\tau}(z_1, z_2; v_1 , v_2)
  & = \frac{1}{z_2 - z_1} \left(\frac{\mathcal{Y}_1(z_1) \mathcal{Y}_2(z_2)}{(z_1 - v_1)(z_2 - v_2)} - 
  \frac{\mathcal{Y}_1(z_2) \mathcal{Y}_2(z_1)}{(z_1 - v_2)(z_2 - v_1)} \right)\\
  & = \frac{1}{z_2 - z_1}
  \left(
  \frac{\mathcal{Y}_1(z_1)}{(z_1 - v_1)} \widetilde{\bm{\tau}}_1(z_2, v_1, v_2) - 
  \frac{\mathcal{Y}_2(z_1)}{(z_1 - v_2)} \widetilde{\bm{\tau}}_2(z_2, v_1, v_2) \right)
\end{split}
\end{equation}
where 
\begin{equation}
 \widetilde{\bm{\tau}}_1(z_2, v_1, v_2) = \frac{\mathcal{Y}_2(z_2)}{(z_2 - v_2)} \quad \textrm{and}  \qquad 
 \widetilde{\bm{\tau}}_2(z_2, v_1, v_2) = \frac{\mathcal{Y}_1(z_2)}{(z_2 - v_1)}  \; .
\end{equation}
\end{subequations}
From this expression, it is easy to extract the residues of \(z_1\)
about one of the Bethe roots.

It is easy to see that the homogeneous limit, that is \(z_2 \to z_1\equiv
w\) gives
\begin{equation}
  \bm{\tau}(w; v_1 , v_2) =
  \frac{1}{(w- v_1)(w- v_2)}\mathcal{W}[\mathcal{Y}_1, \mathcal{Y}_2](w)\; , 
\end{equation}
It is also elementary to see that the functions \(\{\mathcal{Y}_1,
\mathcal{Y}_2\}\) are linearly independent as long as \(v_2 \neq
v_1\), what is guaranteed by the hypothesis that the Bethe roots are
different.


\subsection{n=3} Now we have \(\mathcal{Y}_j\) for \(j=1,2,3\) and three Bethe roots
\(\bm{v} = (v_1, v_2, v_3)\). Moreover, we have the sets \(\bm{v}_1 = (v_2, v_3)\), \(\bm{v}_2 = (v_1, v_3)\)
and \(\bm{v}_3 = (v_1, v_2)\). Then, it is easy to see that
\begin{equation}
\begin{split}
  \mathcal{Y}_j(z)
  & = (\alpha_0(z) + z \alpha_1(z))\sigma_0^{(2)}(\bm{v}_j)   + (\alpha_1(z) + z \alpha_2(z))\sigma_1^{(2)}(\bm{v}_j)  + \\
  & \quad + (\alpha_2(z) + z \alpha_3(z))\sigma_2^{(2)}(\bm{v}_j) \; ,
\end{split}
\end{equation}
with the explicit expressions 
\begin{equation}
  \sigma_0^{(2)}(x, y) = 1\; ,\qquad 
  \sigma_1^{(2)}(x, y) = x + y\; ,\qquad 
  \sigma_2^{(2)}(x, y) = x^2 + xy  + y^2\; . 
\end{equation}

Then, we have
\begin{subequations}
\begin{equation}
  \begin{split}
    \bm{\tau} & (z_1, z_2, z_3 ; v_1 , v_2, v_3)  =
    \frac{1}{(z_2 - z_1)(z_3 - z_1)(z_3 - z_2)}\times \\
    & \times \left[
      \frac{\mathcal{Y}_1(z_1)}{z_1 - v_1}
        \det\begin{bmatrix}
        \frac{\mathcal{Y}_2(z_2)}{(z_2 - v_2)} & \frac{\mathcal{Y}_2(z_3)}{(z_3 - v_2)}\\
        \frac{\mathcal{Y}_3(z_2)}{(z_2 - v_3)} & \frac{\mathcal{Y}_3(z_3)}{(z_3 - v_3)}
        \end{bmatrix}
        -  
      \frac{\mathcal{Y}_2(z_1)}{z_1 - v_2}
        \det\begin{bmatrix}
        \frac{\mathcal{Y}_1(z_2)}{(z_2 - v_1)} & \frac{\mathcal{Y}_1(z_3)}{(z_3 - v_1)}\\
        \frac{\mathcal{Y}_3(z_2)}{(z_2 - v_3)} & \frac{\mathcal{Y}_3(z_3)}{(z_3 - v_3)}
        \end{bmatrix} \right.\\
        & \qquad + \left.
      \frac{\mathcal{Y}_3(z_1)}{z_1 - v_2}
        \det\begin{bmatrix}
        \frac{\mathcal{Y}_1(z_2)}{(z_2 - v_1)} & \frac{\mathcal{Y}_1(z_3)}{(z_3 - v_1)}\\
        \frac{\mathcal{Y}_2(z_2)}{(z_2 - v_2)} & \frac{\mathcal{Y}_2(z_3)}{(z_3 - v_2)}
        \end{bmatrix} 
  \right] \; .
  \end{split}
\end{equation}
We organize this expression as
\begin{equation}
  \begin{split}
    \bm{\tau} (z_1, z_2, z_3 ; v_1 , v_2, v_3)  & =
    \frac{1}{(z_2 - z_1)(z_3 - z_1)}
    \left[
      \frac{\mathcal{Y}_1(z_1)}{z_1 - v_1}
      \left(
      \frac{1}{(z_3 - z_2)}
        \det\begin{bmatrix}
        \frac{\mathcal{Y}_2(z_2)}{(z_2 - v_2)} & \frac{\mathcal{Y}_2(z_3)}{(z_3 - v_2)}\\
        \frac{\mathcal{Y}_3(z_2)}{(z_2 - v_3)} & \frac{\mathcal{Y}_3(z_3)}{(z_3 - v_3)}
        \end{bmatrix}
        \right) - \right. \\
        & - 
      \frac{\mathcal{Y}_2(z_1)}{z_1 - v_2}
      \left(
      \frac{1}{(z_3 - z_2)}
        \det\begin{bmatrix}
        \frac{\mathcal{Y}_1(z_2)}{(z_2 - v_1)} & \frac{\mathcal{Y}_1(z_3)}{(z_3 - v_1)}\\
        \frac{\mathcal{Y}_3(z_2)}{(z_2 - v_3)} & \frac{\mathcal{Y}_3(z_3)}{(z_3 - v_3)}
        \end{bmatrix} \right) \\
        & +
        \left.
      \frac{\mathcal{Y}_3(z_1)}{z_1 - v_2}
      \left(
      \frac{1}{(z_3 - z_2)}
        \det\begin{bmatrix}
        \frac{\mathcal{Y}_1(z_2)}{(z_2 - v_1)} & \frac{\mathcal{Y}_1(z_3)}{(z_3 - v_1)}\\
        \frac{\mathcal{Y}_2(z_2)}{(z_2 - v_2)} & \frac{\mathcal{Y}_2(z_3)}{(z_3 - v_2)}
        \end{bmatrix} 
        \right)
  \right]\; .
  \end{split}
\end{equation}
\end{subequations}

Finally, we conclude that that basis tau functions are
\begin{subequations}
  \begin{equation}
	\tilde{\bm{\tau}}_1(\bm{z}_1) = \frac{1}{(z_3 - z_2)} 
        \det\begin{bmatrix}
        \frac{\mathcal{Y}_2(z_2)}{(z_2 - v_2)} & \frac{\mathcal{Y}_2(z_3)}{(z_3 - v_2)}\\
        \frac{\mathcal{Y}_3(z_2)}{(z_2 - v_3)} & \frac{\mathcal{Y}_3(z_3)}{(z_3 - v_3)}
        \end{bmatrix} 
  \end{equation}
  \begin{equation}
	\tilde{\bm{\tau}}_2(\bm{z}_1) = \frac{1}{(z_3 - z_2)} 
        \det\begin{bmatrix}
        \frac{\mathcal{Y}_1(z_2)}{(z_2 - v_1)} & \frac{\mathcal{Y}_1(z_3)}{(z_3 - v_1)}\\
        \frac{\mathcal{Y}_3(z_2)}{(z_2 - v_3)} & \frac{\mathcal{Y}_3(z_3)}{(z_3 - v_3)}
       \end{bmatrix}
  \end{equation}
  \begin{equation}
	\tilde{\bm{\tau}}_3(\bm{z}_1) = \frac{1}{(z_3 - z_2)} 
        \det\begin{bmatrix}
        \frac{\mathcal{Y}_1(z_2)}{(z_2 - v_1)} & \frac{\mathcal{Y}_1(z_3)}{(z_3 - v_1)}\\
        \frac{\mathcal{Y}_2(z_2)}{(z_2 - v_2)} & \frac{\mathcal{Y}_2(z_3)}{(z_3 - v_2)}
        \end{bmatrix} \; .
  \end{equation}
\end{subequations}

With these expressions, we can consider the case where \(z_1, z_2 z_3
\to w\). Consider first the case \(z_3 \to z_2 = w\), then  
we know that 
\begin{equation}
  \begin{split}
  \tilde{\bm{\tau}}_1(w; v_1 , v_2, v_3) & = \frac{1}{(w - v_2)(w - v_3)}\mathcal{W}[\mathcal{Y}_2, \mathcal{Y}_3](w)\\
  \tilde{\bm{\tau}}_2(w; v_1 , v_2, v_3) & = \frac{1}{(w - v_2)(w - v_3)}\mathcal{W}[\mathcal{Y}_1, \mathcal{Y}_3](w)\\
  \tilde{\bm{\tau}}_3(w; v_1 , v_2, v_3) & = \frac{1}{(w - v_1)(w - v_2)}\mathcal{W}[\mathcal{Y}_1, \mathcal{Y}_2](w)
  \; .
  \end{split}
\end{equation}
and now, it is easy to see that if we also take the limit \(z_1\to w\), we have
\begin{equation}
  \bm{\tau}(w; v_1 , v_2, v_3) = 
  \frac{1}{(w - v_1)(w - v_2)(w - v_3)} \mathcal{W}[\mathcal{Y}_1, \mathcal{Y}_2, \mathcal{Y}_3](w)\; .
\end{equation}

Of course we can keep the calculations for other cases, but now it is obvious that
this explicit analysis become cumbersome very early. It also explains why we
keep our analysis restricted to the Rational cases. 

%%%%%%%%%%%%%%%%%%%%%%%%%%%%%%%%%%%%%%%%%%%%%%%%%%
\section{Baker-Akhiezer function}
%%%%%%%%%%%%%%%%%%%%%%%%%%%%%%%%%%%%%%%%%%%%%%%%%%

This section deals with some properties of the Baker-Akhiezer
functions associated to the tau functions we found above.  Many
aspects of these functions deserve a thorough examination, but here we
investigagte just the most immediate aspects.

%%%%%%%%%%%%%%%%%%%%%%%%%%%%%%%%%%%%%%%%%%%%%%%%%%
\subsection{Integral representation of the tau functions}

We now want to write the Slavnov products we defined above in terms of
the following coordinates
\begin{equation}
  t_p = \frac{1}{p}\sum_{j=1}^n z_j^p\; ,
\end{equation}
the so called Miwa coordinates.

We define
\begin{equation}
  \xi(\bm{t}, \lambda) = \sum_{p=1}^\infty t_p \lambda^p\; , 
\end{equation}
where \(\lambda\) is a complex parameter. Therefore
\begin{equation}
  \begin{split}
    e^{\xi(\bm{t}, \lambda)} & = \exp\left( \sum_{p=1}^\infty t_p \lambda^p \right)
    = \exp\left( \sum_{p=1}^\infty \sum_{j=1}^n \frac{1}{p} z^p \lambda^p \right) \\
    & = \exp\left( \sum_{j=1}^n \sum_{p=1}^\infty  \frac{1}{p} z_j^p \lambda^p \right)
    = \exp\left( - \sum_{j=1}^n \ln( 1 -  z_j \lambda ) \right) \\
    & =  \prod_{j=1}^n \frac{1}{1 -  z_j \lambda} \; .
  \end{split}
\end{equation}
It is easy to see that these functions have simple poles at \( \lambda
= z_j^{-1}\).

Let us now establish an important result,
see ~\cite{Araujo:2021ghu} and references therein. 
\begin{proposition}
The tau function~(\ref{eq:slav-tau-function}) admits the following integral
representation 
\begin{subequations}
\begin{equation}
  \label{eq:tau-function-miwa}
  \bm{\tau}(\bm{t}, \bm{v}) =
    \det_{j,k}\left(
    \oint_{\gamma_k} \frac{d w}{2\pi i} e^{\xi(\bm{t}, w^{-1})} \frac{w^{-j}\mathcal{Y}_k(w)}{w - v_k} \right)\; ,
\end{equation}
where 
\begin{equation}
  \xi(\bm{t}, w^{-1}) = \sum_{p=1}^\infty t_p w^{-p} \quad \Leftrightarrow \quad
  \xi(\bm{z}, w^{-1}) = w^n \prod_{j=1}^n (w - z_j)^{-1} \; . 
\end{equation}
In the above integral, we consider that integration curve \(\gamma_k\)
encloses all poles except the Bethe root \(v_k\).
\end{subequations}
\end{proposition}


\paragraph{\textbf{\emph{Proof}}}
In terms of the \(z\)-coordinates, the tau functions becomes
\begin{equation}
  \label{eq:tau-function}
  \bm{\tau}(\bm{z}, \bm{v}) =
    \det_{j,k}\left(
    \oint_{\gamma_k} \frac{d w}{2\pi i}
    \prod_{l=1}^n \frac{1}{w - z_l} \frac{w^{n-j}\mathcal{Y}_k(w)}{w - v_k} \right) \; . 
\end{equation}
Now, it is easy to see that \(w =0\) is not a pole.  All in all, the
integration curve \(\gamma_k\) encloses points \(\{z_j\}_{j=1}^n\).

Let us define the invertible matrix \(\bm{\mathcal{K}}^T\) by its components
\begin{equation}
    \mathcal{K}_{kj} = 
\oint_{\gamma_k} \frac{d w}{2\pi i} e^{\xi(\bm{z}, w^{-1})} \frac{w^{-j}\mathcal{Y}_k(w)}{w - v_k}\;. 
\end{equation}
Explicitly, we have 
\begin{equation}
  \begin{split}
    \mathcal{K}_{kj}
    = & \oint_{\gamma_k} \frac{d w}{2\pi i} \frac{w^{n-j}}{\prod_{s} (w - z_s)}
    \frac{\mathcal{Y}_k(w)}{w - v_k} = \sum_{\ell =1}^n \frac{z_l^{n-j}}{\prod_{s\neq \ell} (z_\ell - z_s)}
    \frac{\mathcal{Y}_k(z_\ell)}{z_\ell - v_k}\\
    & = \sum_{\ell = 1}^n  \Omega_{k \ell} \left( \frac{z_\ell^{n-j}}{\prod_{s\neq \ell} (z_\ell - z_s)} \right)
  \end{split}
\end{equation}
where in the second line we have used the \(\bm{\Omega}\)-matrix
defined in~(\ref{eq:omega-matrix}).

It is easy to see that
\begin{equation}
  \begin{split}
  \det \bm{\mathcal{K}} & = \det(\bm{\Omega}) \det_{l,j}\left( \frac{z_l^{n-j}}{\prod_{j\neq l} (z_l - z_j)}\right) \\
  & = \det(\bm{\Omega}) \det_{l, j}(z_l^{n-j}) \prod_{\substack{j, l =1 \\ j \neq l}}^n (z_l - z_j)^{-1} \; .
  \end{split}
\end{equation}
We also have the identities 
\begin{subequations}
  \begin{equation}
	\det_{l,j}(z_l^{n-j}) = (-1)^{n(n-1)/2} \Delta(\bm{z})\; ,
  \end{equation}
  and
  \begin{equation}
	\prod_{\substack{j, l = 1 \\ j \neq l}}^n (z_l - z_j) = (-1)^{n(n-1)/2} \Delta(\bm{z})^2\; . 
  \end{equation}
\end{subequations}
Putting all these facts together, we finally see that 
\(\bm{\tau}(\bm{z}, \bm{v}) = \det\bm{\mathcal{K}}\), that
is preciselly the expression~(\ref{eq:slav-tau-function}).\\
\qed

One of the advantages of this representation is
that now it is easier to consider the limit of the infinite chain,
\(n\to \infty\).

%%%%%%%%%%%%%%%%%%%%%%%%%%%%%%%%%%%%%%%%%%%%%%%%%%
\subsection{Baker-Akhiezer  in \(z\)-coordinates}

The Baker-Akhiezer (BA) funcion is defined in terms of the Japanese formula~\cite{Zabrodin2018}
\begin{equation}
  \psi(\bm{t},\bm{v}; \lambda) = e^{\xi(\bm{t}, \lambda)}
  \frac{\bm{\tau}(\bm{t} - [\lambda^{-1}], \bm{v})}{\bm{\tau}(\bm{t}, \bm{v})}
\end{equation}
where 
\begin{equation}
  \bm{t} - [\lambda^{-1}] = 
  \{t_1 - \lambda^{-1}, t_2 - \lambda^{-2}/ 2 ,  t_2 - \lambda^{-3}/3, \dots,  t_p - \lambda^{-p}/p, \dots \}\; .
\end{equation}

Using the expression~(\ref{eq:tau-function-miwa}), we can write the
Baker-Akhiezer function in terms of the \(z\)-coordinates.
Let us first write
\begin{equation}
  t_p - \lambda^{-p}/p = \frac{1}{p} \sum_{j=1}^n z_j^ p - \lambda^{-p}/p \; , 
\end{equation}
therefore
\begin{equation}
  \begin{split}
    e^{\xi(\bm{t} - [\lambda^{-1}], w^{-1})} & = \exp\left[ \sum_{p=1}^\infty \left(t_p  - \frac{\lambda^{-p}}{p}\right) w^{-p} \right]\\
    & = e^{\xi(\bm{t}, w^{-1})} \exp\left[ - \sum_{p=1}^\infty \frac{\lambda^{-p}}{p} w^{-p} \right] \\
    & = e^{\xi(\bm{t}, w^{-1})} \exp\left[ \ln ( 1 - 1 / (\lambda w)) \right]
      = \left( 1 - \frac{1}{\lambda w}\right)  \prod_{j=1}^n \frac{1}{1 -  z_j /w} \\
    & = \frac{w^{n-1}}{\lambda } \left( \lambda w - 1 \right)  \prod_{j=1}^n \frac{1}{w -  z_j} \\
  \end{split}
\end{equation}

Using these expressions, we can write 
\begin{equation}
  \bm{\tau}(\bm{t} - [\lambda^{-1}], \bm{v}) =
  \det_{j,k}\left[
  \oint_{\gamma_k} \frac{dw}{2\pi i} e^{\xi(\bm{t}, w^{-1})} \frac{w^{- j -1} ( \lambda w - 1) \mathcal{Y}_k(w)}{\lambda(w - v_k)} 
  \right]\; .
\end{equation}
that in terms of the \(z\)-coordinates becomes
\begin{equation}
\begin{split}
  \bm{\tau}(\bm{t} - [\lambda^{-1}], \bm{v})
  & = \det_{j,k}\left[\oint_{\gamma_k} \frac{dw}{2\pi i} \prod_{s=1}^n \frac{1}{w - z_s}  
    \frac{( \lambda w - 1) w^{n-j-1}\mathcal{Y}_k(w)}{\lambda(w - v_k)} \right]\\
  & = \det_{j,k}\left[
    \oint_{\gamma_k} \frac{dw}{2\pi i} \prod_{s=1}^n \frac{1}{w - z_s}  
    \frac{w^{n-j-1}\widehat{\mathcal{Y}}_k(w; \lambda)}{(w - v_k)}
    \right]
\end{split}
\end{equation}
where we have defined the function 
\begin{equation}
  \widehat{\mathcal{Y}}_k(w; \lambda)
  = \frac{( \lambda w - 1)}{\lambda} \mathcal{Y}_k(w) \; ,
\end{equation}
From these expressions, it is easy to see that for \(j=n\) the point
\(w = 0\) is a pole; and we must consider it in the integration curve \(\gamma_k\). 

It is convenient to define the components
\begin{equation}
  \begin{split}
    \widetilde{\mathcal{K}}_{kj} = &
    \oint_{\tilde \gamma_k} \frac{dw}{2\pi i} \prod_{s=1}^n \frac{1}{w - z_s}
    \frac{w^{n-j-1}\widehat{\mathcal{Y}}_k(w; \lambda)}{(w - v_k)}
    + \delta_{nj} F_k(\lambda) \\
    & = \sum_{\ell = 1}^n  \widehat{\Omega}_{k \ell} \left( \frac{z_\ell^{n-j}}{\prod_{s\neq \ell} (z_\ell - z_s)} \right)
    + F_k(\lambda)\delta_{nj}  \\
  \end{split}
\end{equation}
where \(\tilde{\gamma}_k\) is a deformation contour that does not pick
the point \(w=0\); and
\begin{equation}
  \widehat{\Omega}_{k\ell}
  = \frac{1}{z_\ell} \frac{\widetilde{\mathcal{Y}}_k(\lambda, z_\ell)}{z_\ell - v_k}
  = \frac{(\lambda z_\ell -1)}{\lambda z_\ell} {\Omega}_{k\ell} 
\; , \qquad 
 F_k(\lambda) = (-1)^{n+1} \frac{\widehat{\mathcal{Y}}_k(0; \lambda)}{v_k }  \prod_{s=1}^n z_s^{-1} \; .
\end{equation}

In order to simplify our analysis, let us define one more matrix
\(\widehat{\mathcal{K}}\) by its components
\begin{equation}
    \widehat{\mathcal{K}}_{kj} = 
    \sum_{\ell = 1}^n  \widehat{\Omega}_{k \ell}
    \left( \frac{z_\ell^{n-j}}{\prod_{s\neq \ell} (z_\ell - z_s)} \right)
    \; .
\end{equation}
We conclude that \(\widetilde{\mathcal{K}}\) can be obtained from 
\(\widehat{\mathcal{K}}\) by adding to each component of the \(n\)-th row
the corresponding component of the vector \(\bm{F} = (F_1, \dots , F_n)\). 
Let us write this operation as 
\begin{equation}
  \widetilde{\mathcal{K}} = \widehat{\mathcal{K}} + \bm{0}_{[ \bm{F} \to n-row]}\; . 
\end{equation}
Where we use the notation \(\mathcal{A}_{[ \bm{F} \to j-row]} \) to
denote the replacement of the \(j-th\) row of the matrix with the
vector \(\bm{F}\).  Above we have considered this operation with the
zero matrix \(\bm{0}\).

By the multilinearity of the determinant, we can write
\begin{equation}
  \det \widetilde{\mathcal{K}} = 
    \det \widehat{\mathcal{K}} + 
    \det \widehat{\mathcal{K}}_{[\bm{F} \to n-row]}\; .
\end{equation}

It is now easy to see that \(\det \widehat{\mathcal{K}}\) is
proportional to the tau functions (Slavnov products), that is
\begin{equation}
    \det \widehat{\mathcal{K}}  = 
    \det(\widehat{\Omega})
    \det_{l,j}
    \left( \frac{z_l^{n-j}}{\prod_{s\neq l} (z_l - z_s)} \right)\\
    = \frac{\det(\widehat{\Omega})}{\Delta(\bm{z})}
\end{equation}
where \(\widehat{\Omega}\) is the matrix formed by the components
\(\widehat{\Omega}_{i,j}\). The above determinant is itself a tau function,
but we can also write is as
\begin{equation}
    \det \widehat{\mathcal{K}} = 
    \frac{\det(\Omega)}{\Delta(\bm{z})}
    \prod_{l=1}^n \left( 1 - \frac{1}{\lambda z_l} \right) 
    = \bm{\tau}(\bm{z}, \bm{v}) \prod_{l=1}^n \left( 1 - \frac{1}{\lambda z_l} \right) \; . 
\end{equation}

Finally, using the elementary symmetric polynomials \(\sigma_p^{(n)}\), we write
\begin{equation}
  \prod_{l=1}^n \left( 1 - \frac{1}{\lambda z_l} \right) = 1 + 
  \sum_{p=1}^n (-1)^p \lambda^{-p} \sigma_p^{(n)}(\bm{z}^{-1})\; , 
\end{equation}
where \( \bm{z}^{-1} = \{z_1^{-1}, z_2^{-1}, \dots , z_n^{-1}\}\). 

Putting all these facts together, we write the Baker-Akhiezer function
as 
\begin{equation}
  \psi(\lambda; \bm{z}, \bm{v}) =
  e^{\xi(\bm{t}, \lambda)}  \left[ 1 + 
  \sum_{p=1}^n (-1)^p \lambda^{-p} \sigma_p^{(n)}(\bm{z}^{-1})
    + \frac{1}{\bm{\tau}(\bm{z}, \bm{v})}
    \det \widehat{\mathcal{K}}_{[\bm{F} \to n-row]}
  \right]\; .
\end{equation}
Next section we find a better expression for the Baker-Akhiezer
function, but from this expression we can see that all the poles
are at the point \(\lambda =0\). 

%%% Maybe I should calculate some examples
%\subsection{Examples} Let us now consider the cases \(n=2, 3\).
%We have calculated the tau functions for these objets. The
%elementary symmetric polynomials are also known. Therefore,
%the only detail we need to complete now is the calculation of
%the matrix \(\widehat{\mathcal{K}}\). 
%
%\subsubsection{n=2} We
%we have \(\mathcal{Y}_j\) for \(j=1,2\) and two Bethe roots


%%%%%%%%%%%%%%%%%%%%%%%%%%%%%%%%%%%%%%%%%%%%%%%%%%
\subsection{Baker-Akhiezer in \(t\)-coordinates}

We now want to consider the above calculations in the Miwa
coordinates. 

We define the matrix \(\check{\mathcal{K}}(\lambda)\) by its components
\begin{equation}
    \check{\mathcal{K}}_{jk} = -
    \oint_{\gamma_k} \frac{dw}{2\pi i}
    e^{\xi(\bm{t}, w^{- 1})} \frac{w^{-j -1 }\mathcal{Y}_k(w)}{w - v_k} \; .
\end{equation}

Therefore, we have that 
\begin{equation}
  \widetilde{\mathcal{K}}_{kj} = 
  \mathcal{K}_{kj}  + \frac{1}{\lambda}
  \check{\mathcal{K}}_{kj}
  \ \Leftrightarrow \
  \widetilde{\mathcal{K}} = 
  \mathcal{K} + \frac{1}{\lambda}
  \check{\mathcal{K}}\; , 
\end{equation}
and since \(\mathcal{K}\) is invertible, we get
\begin{equation}
  \widetilde{\mathcal{K}} = 
  \mathcal{K}\left( \mathbb{1} + 
  \frac{1}{\lambda}\mathcal{K}^{-1}\check{\mathcal{K}} \right)\; .
\end{equation}

Putting all these facts together, we find that the shifted
tau-functions are
\begin{equation}
  \bm{\tau}(\bm{t} - [\lambda^{-1}], \bm{v}) =
  \bm{\tau}(\bm{t}, \bm{v})
  \det \left( \mathbb{1} + \frac{1}{\lambda}
  \mathcal{K}^{-1} \check{\mathcal{K}} \right)\; .
\end{equation}

We finally conclude that the Bakher-Akhiezer function can be written as
\begin{equation}
\label{eq:ba-fredholm}
  \psi(\bm{t},\bm{v}; \lambda) = e^{\xi(\bm{t}, \lambda)}
  \det \left( \mathbb{1} + 
    \frac{1}{\lambda} \mathcal{M}
  \right)\; , 
\end{equation}
where \(\mathcal{M} = \mathcal{K}^{-1} \check{\mathcal{K}}\) is a
finite dimensional matrix. Let us now use the elementary expression
\begin{subequations}
\begin{equation}
  \begin{split}
    \det \left( \mathbb{1} + \frac{1}{\lambda} \mathcal{M}\right)
    & = \exp \left[ \Tr \log \left( \mathbb{1} + \frac{1}{\lambda} \mathcal{M} \right) \right]\\
    & = \prod_{l=1}^n \left( 1 + \frac{\mu_l}{\lambda} \right) \; ,
  \end{split}
\end{equation}
\end{subequations}
where \(\bm{\mu} = \{\mu_l\}_{l=1}^n\) are the eigenvalues of the
matrix \(\mathcal{M}\), and these obviously depend on the Bethe roots
\(\bm{v}\) and parameters \(\bm{t}\). Therefore, we have
\begin{equation}
  \psi(\bm{t},\bm{v}; \lambda) = e^{\xi(\bm{t}, \lambda)}
  \left(1 + \sum_{k=1}^n \frac{\xi_k(\bm{t})}{\lambda^k} 
  \right)\; . 
\end{equation}
where \(\xi_k(\bm{t}) = \sigma_k^{(n)}(\mu_1, \dots, \mu_n)\) and \(\mu_j = \mu_j(\bm{t}, \bm{v})\).

We can also write 
\begin{equation}
  \begin{split}
    \det \left( \mathbb{1} + \frac{1}{\lambda} \mathcal{M}\right)
    & = \exp\left[ \sum_{l=1}^n \ln \left( 1 + \frac{\mu_l}{\lambda} \right)\right]
    = \exp\left[ \sum_{l=1}^n \sum_{k=1}^\infty \frac{(-1)^{k-1}}{k} \left(\frac{\mu_l}{\lambda} \right)^k\right] \\ 
    & = \exp\left[ \sum_{k=1}^\infty \frac{(-1)^{k-1}}{\lambda^k} \gamma_k\right] \; ,
  \end{split}
\end{equation}
where we define the Miwa coordinates \(\gamma_k = \frac{1}{k}\sum_{l=1}^n \mu_l^k \). Consequently
\begin{equation}
    \det \left( \mathbb{1} + \frac{1}{\lambda} \mathcal{M}\right)
     = 1 + \sum_{k\geq 1}\frac{1}{\lambda^k} \sigma_k(\bm{\gamma})\; .
\end{equation}
Putting all these facts together, we can write the functions
\(\xi_k(\bm{t})\) in terms of \(\bm{\gamma}\).  Observe that now the
parameter \(n\) is hidden in the definition of the coordinates
\(\gamma_k\).  One advantgage of this expression is that now we seem
ot have a better setting to consider the thermodynamic limit \(n \to
\infty\) with with \(n/L \to 0\). In fact, the
expression~(\ref{eq:ba-fredholm}) suggests that in this limit, the
Baker-Akhiezer function in the thermodynamics limit can be written as
a Fredholm determinant. We are currently exploring this question and
hope to report new results soon.


%%%%%%%%%%%%%%%%%%%%%%%%%%%%%%%%%%%%%%%%%%%%%%%%%%
\section{Discussion}
%%%%%%%%%%%%%%%%%%%%%%%%%%%%%%%%%%%%%%%%%%%%%%%%%%

In this paper we have discussed some properties of the Slavnov
products and their relation with the tau functions of the KP
hierarchy.

An important initial observation is that the general form of this tau
function is written in terms of some alternant matrix, and this form
is known since (cite papers on the matrix models and so on). But the
functions associated to the current paper is drastically different
from those papers, and do not expand a point in the
Grassmannian. Therefore, it is necessary to better understand some of
these properties. In this paper we start exploring these results.

....  

We have also started our exploration of the Baker-Akhiezer function since it seems to
be the relevant object of our investigation.  

Some future interesting directions. We might want to calculate the
solutions of the KP equation to see what they really mean. Perhaps
consider analitic and numerical calculations, since the calculation of
the determinants become complicated even for small values of \(n\).

More interestingly, one may try to consider the thermodynamic limit of
the Slavnov product and explore what it means from the viewpoint of
the Baker-Akhiezer function.  See the Limit \(L \to \infty\) and \(N
\to \infty\) but sufficiently slower than \(L\).  I think that the
matrix \(\mathcal{M}\) is a trace class operator and we write the BA
function as a Fredholm determinant.

I think that this BA function will be similar to the soliton-like
solution discussed in section 3.6.2 (page 41) of~\cite{Zabrodin2018}.
My functions seem to have a pole of order \(N\) at infinity. And \(N\)
zeros at the Bethe roots.

Finally, we can explore the elliptic case and see if one can see if
there exists any similar relation on the elliptic case.  Elliptic case
- see slavnov, zabrodin's paper



\subsubsection*{Acknowledgments}
I would like to thank...

\printbibliography

\end{document}
