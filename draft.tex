\documentclass[a4paper,11pt]{amsart}

\usepackage{/home/thiago/.config/dot-files/latex/globaldef}
% DEFINITIONS PDFLATEX

%%%%%%%%%%%%%%%%%%%%%%%%
% FONTS AND LANGUAGES
%%%%%%%%%%%%%%%%%%%%%%%% 
\usepackage[T1]{fontenc} % In case I am compiling with Latex. UTF-8 is not necessary anymore
\usepackage[utf8]{inputenc}

%%%%%%%%%%%%%%%%%%%%%%%%
% ADDITIONAL PACKAGES
%%%%%%%%%%%%%%%%%%%%%%%% 
\usepackage{parskip}
\usepackage{tikz} % Drawing package
\usepackage[backend=biber, style=alphabetic, sorting=ydnt]{biblatex}
%\usepackage[backend=biber, style=alphabetic]{biblatex}
\usepackage{listings}
\usepackage{ytableau}


\usepackage{mdframed} % Better than the framed package


\usetikzlibrary{intersections,decorations.text} % this is to make the cover
\usetikzlibrary{matrix, arrows.meta}            % this is to improve diagrams

\setlength\parindent{20pt}

%%%%%%%%%%%%%%%%%%%%%%%%
% METADATA
%%%%%%%%%%%%%%%%%%%%%%%% 

% See options in the official manual
\hypersetup{ 
	pdftitle={integrable systems, programming and so on},
	pdfsubject={High Energy Physics}, 
	pdfauthor={thiago araujo},
	pdfkeywords={gauge; susy; strings; fields; cft; python},
	colorlinks=true,    % false: color frames ; true: color links
    linkcolor=myPurple, % Color of internal links (sections, pages and so on)
    citecolor=myPurple, % color for bibliographical citations
    urlcolor =myPurple, % color for linked URL
    linktocpage=true    % link page to table of contents
}

% Options for listings package I saw the sniipet below here: 
% https://stackoverflow.com/questions/3175105/inserting-code-in-this-latex-document-with-indentation
\lstset{frame=tb,
  language=Python,
  aboveskip=3mm,
  belowskip=3mm,
  showstringspaces=false,
  columns=flexible,
  basicstyle={\small\ttfamily},
  numbers=none,
  numberstyle=\tiny\color{myPurple},
  keywordstyle=\color{myRed},
  commentstyle=\color{myBlue},
  stringstyle=\color{myPurple!80},
  breaklines=true,
  breakatwhitespace=true,
  tabsize=3
}

%%%%%%%%%%%%%%%%%%%%%%%% 
% COLORS
%%%%%%%%%%%%%%%%%%%%%%%% 
% see palette here: https://github.com/enkia/tokyo-night-vscode-theme
\definecolor{myPurple}{RGB}{90, 74, 120}
\definecolor{myBlue}{RGB}{15, 75, 110}
\definecolor{myRed}{RGB}{191,97,106}
\definecolor{myDarkGray}{RGB}{216, 222, 233}
\definecolor{myLightGray}{RGB}{236, 239, 244}

\definecolor{c1}{RGB}{129, 162, 193} % myBlue {15, 75, 110}
\definecolor{c2}{RGB}{216, 222, 233} % myDarkGray
\definecolor{c3}{RGB}{236, 239, 244} % myLightGray
\definecolor{c4}{RGB}{59, 66, 82}
\definecolor{c5}{RGB}{76, 86, 106}

%%%%%%%%%%%%%%%%%%%%%%%%
% THEOREM
%%%%%%%%%%%%%%%%%%%%%%%% 
\newtheorem{theorem}{Theorem}
\newtheorem{corollary}{Corollary}
\newtheorem{proposition}{Proposition}
\newtheorem{conjecture}{Conjecture}
\newtheorem{lemma}{Lemma}
\newtheorem{example}{Example}
\newtheorem{exercise}{Exercise}
\newtheorem{notation}{Notation}
\newtheorem{remark}{Remark}
\newtheorem{definition}{definition}

%%%%%%%%%%%%%%%%%%%%%%%%
% MATH OPERATORS
%%%%%%%%%%%%%%%%%%%%%%%% 
\DeclareMathOperator{\Res}{Res}
\DeclareMathOperator{\Tr}{Tr}

%%%%%%%%%%%%%%%%%%%%%%%%
% MACROS
%%%%%%%%%%%%%%%%%%%%%%%% 
\DeclarePairedDelimiter{\bra}{\langle}{\rvert}
\DeclarePairedDelimiter{\ket}{\lvert}{\rangle}
\DeclarePairedDelimiter{\bbra}{\langle\!\langle}{\rvert}
\DeclarePairedDelimiter{\kket}{\lvert}{\rangle\!\rangle}
\DeclarePairedDelimiterX{\bracket}[2]{\langle}{\rangle}{#1\vert#2}
\DeclarePairedDelimiterX{\bbracket}[2]{\langle\!\langle}{\rangle\!\rangle}{#1\vert#2}


\bibliography{/home/thiago/Sync/wiki/database/bib-database.bib}

%% Extra packages 
\usepackage{amsaddr}                % Affiliation in the first page
\usepackage{slashed}                % Feynman slash notation
\usepackage{simplewick}             % Wick contractions
\usepackage{fontawesome}
\usepackage{tikz-cd}
\usepackage[textwidth=20mm]{todonotes}

\begin{document}

%%%%%%%%%%%%%%%%%%%%%%%%%%%%%%%%%%%%%%%%%%%%%%%%%%
%%%%%%%%%%%%%%%%%%%%%%%%%%%%%%%%%%%%%%%%%%%%%%%%%%

\title{Slavnov products and KP tau functions}

\author{Thiago Araujo}


\address{\noindent 
Instituto de Ciências Exatas, Departamento de Física\\
Universidade Federal Fluminense\\
Rua Des. Ellis Hermydio Figueira, 783, Aterrado\\
27213-145 Volta Redonda, RJ, Brazil
}

\email{\texttt{\href{thiaraujo@id.uff.br}{thiaraujo@id.uff.br}}}

\begin{abstract}
  ...
\end{abstract}

\date{\today}
\keywords{Integrability, Bethe states, Schur, Hall-Littlewood, Toda, KP}
\subjclass[2020]{82B20, 82B23}

\maketitle

\setcounter{tocdepth}{1}
\tableofcontents


%%%%%%%%%%%%%%%%%%%%%%%%%%%%%%%%%%%%%%%%%%%%%%%%%%
\section{Introduction}
%%%%%%%%%%%%%%%%%%%%%%%%%%%%%%%%%%%%%%%%%%%%%%%%%%

%%%%%%%%%%%%%%%%%%%%%%%%%%%%%%%%%%%%%%%%%%%%%%%%%%
\section{Scalar products and determinant formulas}
%%%%%%%%%%%%%%%%%%%%%%%%%%%%%%%%%%%%%%%%%%%%%%%%%%

One important ingredient in the Algebraic Bethe Ansatz is the scalar
product of Bethe states.  The is many important results on this topic,
see~\cite{Korepin:1993kvr} and references therein. Here we focus on
the scalar product between an on-shell and an off-shell Bethe vectors,
the so called \emph{Slavnov product}.

It is known that scalars products in the algebraic Bethe ansatz 
can be written as determinants. In~\cite{Belliard:2019bfz}, the
authors show the reason behind this fact. Let us start
with a short review of this work. 

Let us start with a set of arbitrary complex numbers
\(\bm{u} = \{u_j\}_{j=1}^{n+1}\), and we define \(n+1\) sets
\(\bm{u}_j=\bm{u}\setminus \{u_j\}\) so that we can define
\(n+1\) off-shell Bethe vectors \(\ket{\Psi(\bm{u}_j)}\). 

Consider now a set of Bethe roots \(\bm{v} = \{ v_k \}_{k=1}^n\),
so that we have the on-shell Bethe vectors \(\ket{\Psi(\bm{v})}\). 
In fact, let us consider the action of the transfer matrix \(\mathcal{T}(z)\)
on the dual Bethe vector, that is 
\begin{equation}
\label{eq:eigenvalue-bethe}
\bra{\Psi(\bm{v})}\mathcal{T}(z) = \Lambda(z; \bm{v}) \bra{\Psi(\bm{v})} \; , 
\end{equation}
where \(\Lambda(z; \bm{v}) \) is the eigenvalue of the transfer matrix. 

From these definitions, we now introduce \(n+1\) Slavnov products 
\begin{equation}
  \bm{\zeta}_j = \bracket{\Psi(\bm{v})}{\Psi(\bm{u}_j)}\qquad j =1, \dots, n+1\; .
\end{equation}
We want to show that \(\bm{\zeta}_j\) satisfy a system of linear equations.

In order to see this, we calculate
\begin{equation}
  \bra{\Psi(\bm{v})} {\color{violet}\left(\mathcal{T}(u_j)\ket{\Psi(\bm{u}_j)} \right)} =
  {\color{violet}\left(\bra{\Psi(\bm{v})}\mathcal{T}(u_j)\right)}\ket{\Psi(\bm{u}_j)} 
  \qquad j =1, \dots, n+1\; .
\end{equation}
The right-hand side of this equation can be simplified with the eigenvalue
expression~(\ref{eq:eigenvalue-bethe}), that is
\begin{equation}
  {\color{violet}\left(\bra{\Psi(\bm{v})}\mathcal{T}(u_j)\right)}\ket{\Psi(\bm{u}_j)}  =
\Lambda(u_j; \bm{v}) \bracket{\Psi(\bm{v})}{\Psi(\bm{u}_j)} = 
\Lambda(u_j; \bm{v}) \bm{\zeta}_j\; . 
\end{equation}

In order to simmplify the left-hand side, we expand the action of the transfer matrix on
the off-shell as
\todo{\tiny This point seems to be the most delicate. It does not work for
  elliptic cases}
\begin{equation}
  \mathcal{T}(u_j)\ket{\Psi(\bm{u}_j)}  = \sum_{k=1}^{n+1} L_{jk} \ket{\Psi(\bm{u}_k)} \; ,
\end{equation}
where \(L_{jk}\) are numerical coefficients. Observe that the terms \(L_{jk}\) with
\(j\neq k\) contain the unwanted terms. In the particular case where the algebraic
Bethe equations are satisfied, these unwanted terms vanish and only the diagonal
coefficient survives, that is \(L_{jj}\), and it is equal to the eigenvalue of the
tranfer matrix. 

Therefore, we have
\begin{equation}
  \bra{\Psi(\bm{v})} {\color{violet}\left(\mathcal{T}(u_j)\ket{\Psi(\bm{u}_j)} \right)} =
  \sum_{k=1}^{n+1} L_{jk} \bm{\zeta}_k \; .
\end{equation}

Puttting all these facts together, we get
\begin{equation}
  \sum_{k=1}^{n+1} L_{jk} \bm{\zeta}_k  = 
  \Lambda(u_j; \bar{v}) \bm{\zeta}_j\quad \Rightarrow \quad
  \sum_{k=1}^{n+1} M_{jk} \bm{\zeta}_k = 0 \; ,
\end{equation}
where
\begin{equation}
  M_{jk} = L_{jk} - \delta_{jk}\Lambda(u_j; \bm{v})\; . 
\end{equation}
Here is the system we were looking for. We can write this system in
a matrix form as 
\begin{equation}
  \left(\bm{L} - \bm{\Lambda} \right) \cdot \bm{\zeta} = \bm{0}\; ,
\end{equation}
where
\begin{subequations}
  \begin{equation}
    \bm{L} = 
 \begin{pmatrix}
   L_{1,1} & \cdots & L_{1,n+1}\\
   L_{2,1} & \cdots & L_{2,n+1}\\
   \vdots & \vdots & \vdots\\
   L_{n+1, 1} & \cdots & L_{n+1, n+1}\\
 \end{pmatrix}  \quad , \quad
    \bm{\Lambda} = \mathrm{diag}(\Lambda_1, \Lambda_2, \dots, \Lambda_{n+1})
\end{equation}
where we have written \( \Lambda_j \equiv \Lambda(u_j; \bm{v})\), and also
\begin{equation}
  \bm{\zeta} = (\bm{\zeta}_1, \dots , \bm{\zeta}_{n+1})^T\qquad \, 
  \bm{0} = (0, \dots , 0)^T\; .
\end{equation}
\end{subequations}

The whole point now is that if this system has a solution,
it will be written as minors of the matrix \(M\).


%%%%%%%%%%%%%%%%%%%%%%%%%%%%%%%%%%%%%%%%%%%%%%%%%%
%%%%%%%%%%%%%%%%%%%%%%%%%%%%%%%%%%%%%%%%%%%%%%%%%%

\subsection{Rational models}

We will now consider some concrete models. In the original
work~\cite{Belliard:2019bfz}, the authors consider solutions with a
rational R-mnatrix with periodic boundary conditions. Here we want to
review the main results of this work. 
\todo{\tiny Extend these results to the trigonometric case}

We start with an R-matrix of the form
\begin{equation}
  R(u,v) = \mathbb{1} + g(u,v) \mathcal{P}\; , \qquad
  g(u,v) = \frac{c}{u - v}
\end{equation}
where \(c\) is a constant. Moreover \(\mathbb{1}\) is the identity operator and
\(\mathcal{P}\) is the permutation operator.

In the models with rational R-matrix, the eigenvalues of the transfer matrix
have the form 
\begin{equation}
  \Lambda(z, \bm{v}) = g(z, \bm{v}) \mathcal{Y}(z; \bm{v})\; ,
\end{equation}
where the function \(\mathcal{Y}(z; \bm{v})\) is symmetric over the
Bethe roots \(\bm{v}\), and linearly depends on every
\(v_j\). Generically, it is written as
\begin{equation}
  \mathcal{Y}(z; \bm{v}) = \sum_{p=0}^n \alpha_p(z) \sigma_p^{(n)}(\bm{v})\; , 
\end{equation}
where \(\sigma_p^{(n)}(\bm{v})\) are elementary symmetric polynomials
in \(\bm{v}\) and where \(\alpha_p(z)\) are free functional
parameters. In fact, in the XXX discussed in~\cite{Belliard:2019bfz},
these functions all \(\alpha_p(z)\) are polynomials of degree \(2n\). 
The important point for us is that \(\alpha_p(z)\) is not
singular when we consider the limit \(z \to v_j \in \bm{v}\),
so does the function \(\mathcal{Y}(z; \bm{v})\). 

Furthermore, we have
\begin{equation}
  g(z, \bm{v}) = \prod_{v_i\in \bm{v}} g(z; v_i)\; , 
\end{equation}
and that means that each Bethe root \( v_i \in \bm{v}\) of this expression
is a pole of \(g(z;\bm{v})\). Therefore, if \(\gamma_j\) is a contour
containing the point \(v_j\), we have 
\begin{equation}
  \frac{1}{2 \pi i}\oint_{\gamma_j} \mathrm{d}z g(z; \bm{v}) = \prod_{v_i \in \bm{v}\setminus \{u_j\}} g(z, v_i)\; .
\end{equation}

The coefficients \(L_{jk}\) are given by \todo{\tiny Check references
  1, 2 of the paper to see the trigonometric case.}
\begin{equation}
  L_{jk} = g(u_k, \bm{u}_k) \mathcal{Y}(u_k; \bm{u}_j)
\end{equation}
and from this expression one can calculate the matrix \(\bm{M}\).
Finally, the authors also show that if \(\det (\bm{M}) = 0 \) and
\(\mathrm{rank}(\bm{M}) = n\), then, the Slavnov products are given by
\begin{equation}
 \bm{\zeta}_\ell = \phi(\bm{v}) \tilde{\Delta}(\bm{u}_\ell)\hat{\Omega}_\ell\; ,
\end{equation}
where \(\phi(\bm{v})\) is a function of the Bethe roots, and also
\begin{equation}
  \tilde{\Delta}(\bm{u}_\ell) = \prod_{\substack{u_j, u_j \in \bm{u}_\ell \\ j>k}} \frac{c}{u_j - u_k}\; . 
\end{equation}
Therefore, we can absorb the product of the constant \(c\) into
a unique constant \(c_0\) so that 
\begin{equation}
  \frac{\tilde{\Delta}(\bm{u}_\ell)}{c_0} = \prod_{\substack{u_j, u_j \in \bm{u}_\ell \\ j>k}} \frac{1}{u_j - u_k}\; , 
\end{equation}
that is the inverse of the Vandermonde determinant
\(\Delta(\bm{u}_\ell)\).
Finally, we have  that \(\hat{\Omega}_\ell\) are minors of the matrix \(\bm{\Omega}\) with the \(\ell\)-th
column excluded, that is 
\begin{equation}
 \hat{\Omega}_\ell = \det_{k\neq \ell} \Omega_{jk}\; ,
\end{equation}
where
\begin{equation}
  \Omega_{jk} = g(u_k, v_j) \mathcal{Y}(u_k; \{u_k, \bm{v}_j\})
  \qquad j= 1, \dots, n\; , \quad k =1, \dots, n+1
\end{equation}

All in all, we have an expression of the
form
\begin{equation}
  \bm{\tau}_\ell(\bm{u}_\ell; \bm{v}) \equiv \frac{1}{c_0}\frac{\bm{\zeta}_\ell}{\phi(\bm{v})}
  = \frac{\det_{k\neq \ell} \Omega_{jk}}{\Delta(\bm{u}_\ell)}
\end{equation}

Let us now simplify these thing a little bit. First of all, we need to
observe that we can define \(n\) functions \(\Omega_j(z)\) of the form 
\begin{equation}
\begin{split}
  \Omega_{j}(z; \bm{v}) & = g(z, v_j) \mathcal{Y}(z; \{z, \bm{v}_j\})\qquad j =1, \dots, n\\
   & = \frac{c}{z - v_j} \mathcal{Y}(z; \{z, \bm{v}_j\})\; .
\end{split}
\end{equation}
Moreover, we have the expansion 
\begin{equation}
  \mathcal{Y}(z; \{z,\bm{v}_j\}) = \sum_{p=0}^n \alpha_p(z) \sigma_p^{(n)}(\{z,\bm{v}_j\})\; , 
\end{equation}
and using that the elementary symmetric polynomials decompose as
\begin{equation}
  \sigma_p^{(n)}(\{z,\bm{v}_j\}) = \sigma_p^{(n-1)}(\bm{v}_j) + z \sigma_{p-1}^{(n-1)}(\bm{v}_j) \; , 
\end{equation}
Therefore,
\begin{equation}
  \label{eq:y-functions-alpha}
  \mathcal{Y}_j(z)  = 
\sum_{p=0}^{n} \alpha_p(z)
  \left(\sigma_p^{(n-1)}(\bm{v}_j) + z \sigma_{p-1}^{(n-1)}(\bm{v}_j) \right) 
\end{equation}
where we have denoted \( \mathcal{Y}(z; \{z,\bm{v}_j\}) \equiv \mathcal{Y}_j(z)\). 
Observe that what defines the functions \(\mathcal{Y}_j(z)\) is the absence of the
\(j\)-th Bethe root \(v_j\).

Using that the elementary symmetric polynomials also satisfy \(\sigma_n^{(n-1)}(\bm{v}_j)=0\)
and \(\sigma_{-1}^{(n)}(\bm{v}_j)= 0\), we can write
\begin{equation}
  \label{eq:y-functions}
  \mathcal{Y}_j(z)  = 
  \sum_{p=0}^{n-1} 
   \left(\alpha_p(z) + z \alpha_{p+1}(z) \right) \sigma_p^{(n-1)}(\bm{v}_j) 
\end{equation}

Let us write the \(p\) functions \(\sigma_p\) in terms of a linearly
\todo{Not sure if I need this!!!}
independent basis \(\{\phi_j\}_{j=1}^n\). In this case, we have that the functions
\(\mathcal{Y}_j(z)\) are also linearly independent and have the expansion
\begin{equation}
  \mathcal{Y}_j(z) = \sum_{k=1}^n \rho_k(\bm{v}_j) \phi_k(z)\; ,
\end{equation}
where the coefficients \( \rho_k(\bm{v}_j)\) depend on the Bethe roots \(\bm{v}_j\). 
The functional structure of the coefficients \(\rho_k\), when written, in terms of the
Bethe roots are the same.  

For example, in the case of the XXX model, \(\alpha_p(z)\sim z^{2n} + \cdots\), so
that one can take \(\phi_j \sim z^{2j} + \cdots\), and the coefficients depend on the
Bethe roots and inhomogeneities.

We finally have 
\begin{equation}
  \Omega_{j}(z; \bm{v}) = -\frac{c}{v_j - z} \mathcal{Y}_j(z) \; .
\end{equation}
where we have denoted \( \mathcal{Y}(z; \{z,\bm{v}_j\}) \equiv \mathcal{Y}_j(z)\).
It is also convenient to consider \(c=1\). 

%Finally, let us now consider this function in the neighborhood of the Bether root \(v_j\). We find 
%\begin{equation}
%  \Omega_j(z; \bm{v}) \simeq \frac{c}{z - v_j}
%\mathcal{Y}(v_j; \bm{v})\qquad j =1, \dots, n\; .
%\end{equation}



%%%%%%%%%%%%%%%%%%%%%%%%%%%%%%%%%%%%%%%%%%%%%%%%%%
%%%%%%%%%%%%%%%%%%%%%%%%%%%%%%%%%%%%%%%%%%%%%%%%%%

\subsection{Tau functions}

Therefore, we can write the matrix \(\bm{\Omega}\) as 
\begin{equation}
  \bm{\Omega} =
  \begin{pmatrix}
    \Omega_1(u_1) & \dots & \Omega_1(u_n) & \Omega_1(u_{n+1}) \\
    \Omega_2(u_1) & \dots & \Omega_2(u_n) & \Omega_2(u_{n+1}) \\
    \vdots & \vdots & \vdots & \vdots \\
    \Omega_n(u_1) & \dots & \Omega_n(u_n) & \Omega_n(u_{n+1})
  \end{pmatrix}
\end{equation}
and we also define the \(n\times n\) matrices \(\bm\Omega^{(\ell)}(\bm{u}_\ell; \bm{v})\),
\(\ell = 1, \dots, n+1\), removing the \(\ell\)-th column of
\(\bm{\Omega}\), that is
\begin{equation}
  \bm{\Omega}^{(\ell)} =
  \begin{pmatrix}
    \Omega_1(u_1) & \dots & \Omega_1(u_{\ell-1}) & \Omega_1(u_{\ell+1})&\dots & \Omega_1(u_{n+1}) \\
    \vdots & \dots & \vdots & \vdots & \dots & \vdots \\
    \Omega_n(u_1) & \dots & \Omega_n(u_{\ell-1}) & \Omega_n(u_{\ell+1})&\dots & \Omega_n(u_{n+1})\; .
  \end{pmatrix}
\end{equation}

Putting these facts together, we write the functions
\begin{equation}
\label{eq:slavnov-tau-functions}
  \bm{\tau}_\ell(\bm{u}_\ell; \bm{v}) = \frac{\det [\Omega_j((\bm{u}_\ell)_k)]|_{j,k=1}^n }{\Delta(\bm{u}_\ell)} 
\end{equation}
where \((\bm{u}_\ell)_k \) is the k-th component of the vector
\(\bm{u}_\ell\). These functions are tau functions of the KP
hierarchy.  We have also omitted the dependence on the Bethe roots
\(\bm{v}\). The most important aspect now is that the poles of this
tau function are exactly the Bethe roots of the integrable model.

It is well documented that the functions defined as
in~(\ref{eq:slavnov-tau-functions}) are tau functions of the
KP-hierachy, see~\cite{Araujo:2021ghu} and references therein, in
particular~\cite{Kharchev:1991cy}.

In fact, we can focus all our attention on a particular function. Therefore,
for a fixed \(\ell\), we define the vector \( \bm{z} \equiv \bm{u}_\ell\).
Consequently, 
\begin{equation}
\begin{split}
  \bm{\tau}(\bm{z}, \bm{v})
  & = \frac{1}{\Delta(\bm{z})} \det[\Omega_j(z_k) ]_{j,k=1}^n \\
  & = \frac{1}{\Delta(\bm{z})} \det\left[\frac{\mathcal{Y}_j(z_k)}{z_k - v_j} \right]_{j,k=1}^n  \\
\end{split}
\end{equation}
All these tau functions have the form of an
\emph{alternant determinant} \footnote{see definition~(\ref{def:alt-determinant})
in the appendix~\ref{app:mathematical}.}

\subsection{Pole structure}

We now want to analyse the pole structure of
\(\Delta(\bm{z})\tau(\bm{z}, \bm{v})\).  First of all, one should
observe that the Bethe roots \(\bm{v}\) appear as poles of the
coefficients \(\bm{z}\). Moreover, lets us consideer here that the
Bethe roots are not degenerate, that is, if \(j \neq k\), then \(v_j
\neq v_k\). In that case, the poles \(v_j\) are simple poles.

Let us now consider a particular coefficient \(\bm{z}\), say \(z_l\), and
make a Laplace expansion in the column \(l\), that is
\begin{equation}
\begin{split}
  \bm{\tau}(\bm{z}, \bm{v})
  & = \frac{1}{\Delta(\bm{z})} \det[\bm{\Omega}(\bm{z}) ] \\
  & = \frac{1}{\Delta(\bm{z})} \sum_{j=1}^n (-1)^{j + l} \frac{\mathcal{Y}_j(z_l)}{z_l - v_j  } 
  \det[\hat{\Omega}_{j, l}]\; ,
\end{split}
\end{equation}
where \(\hat{\Omega}_{j,l} \) is the \((j, l)\) minor of \(\bm{\Omega}\).

Now, we can extract the residue of the tau function for \(z_l\) at the point \(v_j\), that is
\begin{equation}
  \Res_{(z_l, v_j)}(\Delta(\bm{z})\tau(\bm{z}, \bm{v})) = (-1)^{j+l} \mathcal{Y}_j(v_j) \det[\hat{\Omega}_{j, l}]
\end{equation}

Let us now decompose the Vandermonde determinant as
\begin{equation}
  \Delta(\bm{z}) = \prod_{j>k}(z_j - z_k) = \Delta(\bm{z}_l) \prod_{r<l}(z_l - z_r) \prod_{s>l}(z_s - z_l)\; ,
\end{equation}
where \(\bm{z}_l = \bm{z}\setminus \{z_l\} = (z_1, \dots, z_{l-1}, z_{l+1}, \dots, z_n) \).
\todo[inline]{I wand to verify that this is correct}
We can define the function
\begin{equation}
  \Xi_l(z_l,\bm{z}_l) = \prod_{r<l}(z_l - z_r) \prod_{r>l}(z_r - z_l)\; , 
\end{equation}
then 
\begin{equation}
  \Delta(\bm{z}) = \Xi_l(z_l, \bm{z}_l) \Delta(\bm{z}_l) \; .
\end{equation}

Let us now write
\begin{equation}
  \bm{\tau}(\bm{z}, \bm{v})
   = \frac{1}{\Xi_l(z_l, \bm{z}_l)} \sum_{j=1}^n (-1)^{j + l} \frac{\mathcal{Y}_j(z_l)}{z_l - v_j} 
\left(\frac{1}{\Delta(\bm{z}_l)} \det[\hat{\Omega}_{j, l}]\right)\; .
\end{equation}

When we define the set of functions
\begin{equation}
  \begin{split}
  \hat{\bm{\Omega}}_{(j)}(z) & = \left(\Omega_1(z), \dots, \Omega_{j-1}(z), \Omega_{j+1}(z), \dots, \Omega_n(z) \right)\; ,
  \end{split}
\end{equation}
we can simplify the tau function expansion even further. Then
\begin{equation}
\begin{split}
  \bm{\tau}(\bm{z}, \bm{v})
  & = \frac{1}{\Xi_l(z_l, \bm{z}_l)} \sum_{j=1}^n (-1)^{j + l} \frac{\mathcal{Y}_j(z_l)}{z_l - v_j} 
    \left(\frac{1}{\Delta(\bm{z}_l)} \det[\hat{\bm{\Omega}}_{(j) r}(\bm{z}_{(l) s} ]_{r, s=1}^n\right)\\ 
  & = \frac{1}{\Xi_l(z_l, \bm{z}_l)} \sum_{j=1}^n (-1)^{j + l} \frac{\mathcal{Y}_j(z_l)}{z_l - v_j} 
    \tilde{\bm{\tau}}_j(\bm{z}_l, \bm{v})\; ,
\end{split}
\end{equation}
where 
\begin{equation}
  \label{eq:basis-tau}
    \tilde{\bm{\tau}}_j(\bm{z}_l, \bm{v}) = 
\frac{1}{\Delta(\bm{z}_l)} \det[\hat{\bm{\Omega}}_{(j)}(\bm{z}_{(l)} ]\; .
\end{equation}
The curious aspect of this expression is that it easy to see that each term  
\(\tilde{\bm{\tau}}_j(\bm{z}_l, \bm{v})\) is a tau function itself, and that
it acts as a basis for the Slavnov's product. 
Additionally, these basis tau functions \(\tilde{\tau}_j(\bm{z}_l,
\bm{v})\) are, by construction, completely independent of \(z_l\).

Finally, we have
\begin{equation}
  \Res_{(z_l, v_j)}(\tau(\bm{z}, \bm{v}))
  = \frac{(-1)^{j + l} \mathcal{Y}_j(v_j)}{\Xi_l(v_j, \bm{z}_l)}
  \tilde{\bm{\tau}}_j(\bm{z}_l, \bm{v})
\end{equation}

In order to avoid cluttering, let us order the parameters \(\bm{z}\) and
consider these points close to the corresponding \(\bm{v}\), that is,
we consider just the residues \(z_j \to v_j\), then
\begin{subequations}
\begin{equation}
  \Res_{(z_j, v_j)}(\tau(\bm{z}, \bm{v}))
  = \frac{\mathcal{Y}_j(v_j)}{\Xi_j(v_j, \bm{z}_j)}
  \tilde{\bm{\tau}}_j(\bm{z}_j, \bm{v})
\end{equation}
or yet 
\begin{equation}
  \tilde{\bm{\tau}}_j(\bm{z}_j, \bm{v})
  = \frac{\Xi_j(v_j, \bm{z}_j)}{\mathcal{Y}_j(v_j)}
  \Res_{(z_j, v_j)}(\tau(\bm{z}, \bm{v}))
\end{equation}
\end{subequations}
Evidently, one can repeat the same ideas and take \(z_k \in \bm{z}_j\)
close to \(v_k \in \bm{v}_j\). 

\subsection{Examples} Let us now consider the cases \(n=2, 3\) (the case \(n=1\) is trivial)
to gain some understanding of the problem.

\subsubsection{n=2} In this case, we have \(\mathcal{Y}_j\) for \(j=1,2\) and two Bethe roots
\(\bm{v} = (v_1, v_2)\). Moreover, we have the two sets \(\bm{v}_1 =
\{v_2\}\) and \(\bm{v}_2 = \{v_1\}\). From (\ref{eq:y-functions}), we have
\begin{subequations}
\begin{equation}
  \begin{split}
    \mathcal{Y}_j(z) =
    \alpha_0(z) \sigma_0^{(1)}(\bm{v}_j) +  
    \alpha_1(z)( \sigma_1^{(1)}(\bm{v}_j) + z \sigma_0^{(1)}(\bm{v}_j)) +
    \alpha_2(z) z \sigma_1^{(1)}(\bm{v}_j) \; , 
  \end{split}
\end{equation}
finally, the explicit formulas for the elementary symmetric polynomials are \(\sigma_0^{(1)}(x) =1\)
and \(\sigma_1^{(1)}(x) = x\). 

Therefore
\begin{equation}
  \begin{split}
    \mathcal{Y}_1(z) & = \alpha_0(z) + z \alpha_1(z)  + v_2 \left( \alpha_1(z) + z \alpha_2(z) \right) \\ 
    \mathcal{Y}_2(z) & = \alpha_0(z) + z \alpha_1(z)  + v_1 \left( \alpha_1(z) + z \alpha_2(z) \right) ; .
  \end{split}
\end{equation}
Thefore, we find
\begin{equation}
  \bm{\tau}(z_1, z_2; v_1 , v_2) = \frac{1}{z_2 - z_1}
  \left(\frac{\mathcal{Y}_1(z_1) \mathcal{Y}_2(z_2)}{(z_1 - v_1)(z_2 - v_2)} - 
  \frac{\mathcal{Y}_1(z_2) \mathcal{Y}_2(z_1)}{(z_1 - v_2)(z_2 - v_1)} 
  \right)
\end{equation}
\end{subequations}
One interesting case happens for \(z_2 \to z_1\equiv w\), in this case
it is easy to show that
\begin{equation}
  \bm{\tau}(w; v_1 , v_2) =
  \frac{1}{(w- v_1)(w- v_2)}\mathcal{W}[\mathcal{Y}_1, \mathcal{Y}_2](w)\; , 
\end{equation}
where \(\mathcal{W}[\mathcal{Y}_1, \dots, \mathcal{Y}_n](w)\) denotes the
Wronskian of the functions \(\{\mathcal{Y}_j(w)\}_{j=1}^n\), in our
case, we have \(n=2\). In this case, one can observe that these
functions are linearly independent as long as \(v_2 \neq v_1\), what
is guaranteed by the hypothesis that the Bethe roots are different. 



\subsubsection{n=3} Now we have \(\mathcal{Y}_j\) for \(j=1,2,3\) and three Bethe roots
\(\bm{v} = (v_1, v_2, v_3)\). Moreover, we have the sets \(\bm{v}_1 = (v_2, v_3)\), \(\bm{v}_2 = (v_1, v_3)\)
and \(\bm{v}_3 = (v_1, v_2)\). Then, it is easy to see that
\begin{equation}
\begin{split}
  \mathcal{Y}_j(z)
  & = (\alpha_0(z) + z \alpha_1(z))\sigma_0^{(2)}(\bm{v}_j)   + (\alpha_1(z) + z \alpha_2(z))\sigma_1^{(2)}(\bm{v}_j)  + \\
  & \quad + (\alpha_2(z) + z \alpha_3(z))\sigma_2^{(2)}(\bm{v}_j) 
\end{split}
\end{equation}
Moreover, one can also use the explicit expressions 
\begin{equation}
  \sigma_0^{(2)}(x, y) = 1\; ,\qquad 
  \sigma_1^{(2)}(x, y) = x + y\; ,\qquad 
  \sigma_2^{(2)}(x, y) = x^2 + xy  + y^2\; , 
\end{equation}
to determine the functions \(\mathcal{Y}_j(z)\) for \(j=1,2,3\).

Then, we have
\begin{subequations}
\begin{equation}
  \begin{split}
    \bm{\tau} & (z_1, z_2, z_3 ; v_1 , v_2, v_3)  =
    \frac{1}{(z_2 - z_1)(z_3 - z_1)(z_3 - z_2)}\times \\
    & \times \left[
      \frac{\mathcal{Y}_1(z_1)}{z_1 - v_1}
        \det\begin{bmatrix}
        \frac{\mathcal{Y}_2(z_2)}{(z_2 - v_2)} & \frac{\mathcal{Y}_2(z_3)}{(z_3 - v_2)}\\
        \frac{\mathcal{Y}_3(z_2)}{(z_2 - v_3)} & \frac{\mathcal{Y}_3(z_3)}{(z_3 - v_3)}
        \end{bmatrix}
        -  
      \frac{\mathcal{Y}_2(z_1)}{z_1 - v_2}
        \det\begin{bmatrix}
        \frac{\mathcal{Y}_1(z_2)}{(z_2 - v_1)} & \frac{\mathcal{Y}_1(z_3)}{(z_3 - v_1)}\\
        \frac{\mathcal{Y}_3(z_2)}{(z_2 - v_3)} & \frac{\mathcal{Y}_3(z_3)}{(z_3 - v_3)}
        \end{bmatrix} \right.\\
        & \qquad + \left.
      \frac{\mathcal{Y}_3(z_1)}{z_1 - v_2}
        \det\begin{bmatrix}
        \frac{\mathcal{Y}_1(z_2)}{(z_2 - v_1)} & \frac{\mathcal{Y}_1(z_3)}{(z_3 - v_1)}\\
        \frac{\mathcal{Y}_2(z_2)}{(z_2 - v_2)} & \frac{\mathcal{Y}_2(z_3)}{(z_3 - v_2)}
        \end{bmatrix} 
  \right]
  \end{split}
\end{equation}
That we can organize as
\begin{equation}
  \begin{split}
    \bm{\tau} (z_1, z_2, z_3 ; v_1 , v_2, v_3)  & =
    \frac{1}{(z_2 - z_1)(z_3 - z_1)}
    \left[
      \frac{\mathcal{Y}_1(z_1)}{z_1 - v_1}
      \left(
      \frac{1}{(z_3 - z_2)}
        \det\begin{bmatrix}
        \frac{\mathcal{Y}_2(z_2)}{(z_2 - v_2)} & \frac{\mathcal{Y}_2(z_3)}{(z_3 - v_2)}\\
        \frac{\mathcal{Y}_3(z_2)}{(z_2 - v_3)} & \frac{\mathcal{Y}_3(z_3)}{(z_3 - v_3)}
        \end{bmatrix}
        \right) - \right. \\
        & - 
      \frac{\mathcal{Y}_2(z_1)}{z_1 - v_2}
      \left(
      \frac{1}{(z_3 - z_2)}
        \det\begin{bmatrix}
        \frac{\mathcal{Y}_1(z_2)}{(z_2 - v_1)} & \frac{\mathcal{Y}_1(z_3)}{(z_3 - v_1)}\\
        \frac{\mathcal{Y}_3(z_2)}{(z_2 - v_3)} & \frac{\mathcal{Y}_3(z_3)}{(z_3 - v_3)}
        \end{bmatrix} \right) \\
        & +
        \left.
      \frac{\mathcal{Y}_3(z_1)}{z_1 - v_2}
      \left(
      \frac{1}{(z_3 - z_2)}
        \det\begin{bmatrix}
        \frac{\mathcal{Y}_1(z_2)}{(z_2 - v_1)} & \frac{\mathcal{Y}_1(z_3)}{(z_3 - v_1)}\\
        \frac{\mathcal{Y}_2(z_2)}{(z_2 - v_2)} & \frac{\mathcal{Y}_2(z_3)}{(z_3 - v_2)}
        \end{bmatrix} 
        \right)
  \right]\; .
  \end{split}
\end{equation}
\end{subequations}

Finally, we conclude that
\begin{subequations}
  \begin{equation}
	\tilde{\bm{\tau}}_1(\bm{z}_1) = \frac{1}{(z_3 - z_2)} 
        \det\begin{bmatrix}
        \frac{\mathcal{Y}_2(z_2)}{(z_2 - v_2)} & \frac{\mathcal{Y}_2(z_3)}{(z_3 - v_2)}\\
        \frac{\mathcal{Y}_3(z_2)}{(z_2 - v_3)} & \frac{\mathcal{Y}_3(z_3)}{(z_3 - v_3)}
        \end{bmatrix} 
  \end{equation}
  \begin{equation}
	\tilde{\bm{\tau}}_2(\bm{z}_1) = \frac{1}{(z_3 - z_2)} 
        \det\begin{bmatrix}
        \frac{\mathcal{Y}_1(z_2)}{(z_2 - v_1)} & \frac{\mathcal{Y}_1(z_3)}{(z_3 - v_1)}\\
        \frac{\mathcal{Y}_3(z_2)}{(z_2 - v_3)} & \frac{\mathcal{Y}_3(z_3)}{(z_3 - v_3)}
       \end{bmatrix}
  \end{equation}
  \begin{equation}
	\tilde{\bm{\tau}}_3(\bm{z}_1) = \frac{1}{(z_3 - z_2)} 
        \det\begin{bmatrix}
        \frac{\mathcal{Y}_1(z_2)}{(z_2 - v_1)} & \frac{\mathcal{Y}_1(z_3)}{(z_3 - v_1)}\\
        \frac{\mathcal{Y}_2(z_2)}{(z_2 - v_2)} & \frac{\mathcal{Y}_2(z_3)}{(z_3 - v_2)}
        \end{bmatrix} 
  \end{equation}
\end{subequations}

With these expressions, we can consider the case where \(z_1, z_2 z_3
\to w\). Consider first the case \(z_3 \to z_2 = w\), then  
we know that 
\begin{equation}
  \begin{split}
  \tilde{\bm{\tau}}_1(w; v_1 , v_2, v_3) & = \frac{1}{(w - v_2)(w - v_3)}\mathcal{W}[\mathcal{Y}_2, \mathcal{Y}_3](w)\\
  \tilde{\bm{\tau}}_2(w; v_1 , v_2, v_3) & = \frac{1}{(w - v_2)(w - v_3)}\mathcal{W}[\mathcal{Y}_1, \mathcal{Y}_3](w)\\
  \tilde{\bm{\tau}}_3(w; v_1 , v_2, v_3) & = \frac{1}{(w - v_1)(w - v_2)}\mathcal{W}[\mathcal{Y}_1, \mathcal{Y}_2](w)
  \end{split}\; .
\end{equation}
and now, it is easy to see that if we also take the limit \(z_1\to w\), we have
\begin{equation}
  \bm{\tau}(w; v_1 , v_2, v_3)
  \frac{1}{(w - v_1)(w - v_2)(w - v_3)} \mathcal{W}[\mathcal{Y}_1, \mathcal{Y}_2, \mathcal{Y}_3](w)\; .
\end{equation}

\subsubsection{Homogeneous limit and Wronskian formula}

We now want to consider the limit where \(z_k\to z_1\), \(k=2,\dots, n\). 
Let us first consider the case \(z_2 \to z_1\), then
\begin{subequations}
\begin{equation}
  \bm{\tau}(\bm{z}, \bm{v}) = \frac{1}{\Delta(\bm{z})}
  \det
  \begin{pmatrix}
    \frac{\mathcal{Y}_1(z_1)}{z_1 - v_1}  & \frac{\mathcal{Y}_1(z_1)}{z_1 - v_1} + (z_2 - z_1) \frac{\mathcal{Y}_1^{(1)}(z_1)}{z_1 - v_1}
    + \mathcal{O}(\delta^2) &
    \frac{\mathcal{Y}_1(z_3)}{z_3 - v_1} & \dots & \frac{\mathcal{Y}_1(z_n)}{z_n - v_1}\\
    \frac{\mathcal{Y}_2(z_1)}{z_1 - v_2} & \frac{\mathcal{Y}_2(z_1)}{z_1 - v_2} + (z_2 - z_1) \frac{\mathcal{Y}_2^{(1)}(z_1)}{z_1 - v_2} 
    + \mathcal{O}(\delta^2) &
    \frac{\mathcal{Y}_2(z_3)}{z_3 - v_2} &  \dots & \frac{\mathcal{Y}_2(z_n)}{z_n - v_2}\\
    &  \vdots & & \\
    \frac{\mathcal{Y}_n(z_1)}{z_1 - v_n} & \frac{\mathcal{Y}_n(z_1)}{z_1 - v_n} + (z_2 - z_1) \frac{\mathcal{Y}_n^{(1)}(z_1)}{z_1 - v_n} 
    + \mathcal{O}(\delta^2) &
    \frac{\mathcal{Y}_n(z_3)}{z_3 - v_n} &  \dots & \frac{\mathcal{Y}_n(z_n)}{z_n - v_n}
  \end{pmatrix}
\end{equation}
where we have written \(z_2 - z_1 = \delta \to 0\). Moreover, we write
\(\mathcal{Y}^{(n)}(z)\) as the nth derivative with respect its
argument. Now one can immediatelly see that the second column is the
first column plus terms multiplied by the factor \(\delta = z_2 -
z_1\). Using elementary operations on the column, we find
\begin{equation}
  \bm{\tau}_h(\bm{z}, \bm{v}) = \lim_{z_2 \to z_1}\frac{(z_2 - z_1)}{\Delta(\bm{z})}
  \det
  \begin{pmatrix}
    \frac{\mathcal{Y}_1(z_1)}{z_1 - v_1}  & \frac{\mathcal{Y}_1^{(1)}(z_1)}{z_1 - v_1} &
    \frac{\mathcal{Y}_1(z_3)}{z_3 - v_1} & \dots & \frac{\mathcal{Y}_1(z_n)}{z_n - v_1}\\
    \frac{\mathcal{Y}_2(z_1)}{z_1 - v_2} & \frac{\mathcal{Y}_2^{(1)}(z_1)}{z_1 - v_2} &
    \frac{\mathcal{Y}_2(z_3)}{z_3 - v_2} &  \dots & \frac{\mathcal{Y}_2(z_n)}{z_n - v_2}\\
    &  \vdots & & \\
    \frac{\mathcal{Y}_n(z_1)}{z_1 - v_n} & \frac{\mathcal{Y}_n^{(1)}(z_1)}{z_1 - v_n} &
    \frac{\mathcal{Y}_n(z_3)}{z_3 - v_n} &  \dots & \frac{\mathcal{Y}_n(z_n)}{z_n - v_n}
  \end{pmatrix}\; .
\end{equation}
We can now repeat the same ideas for \(z_3\to z_1=z_2\). We find that
the third column is the first plus the second and the derivatives
multiplied by the factor \(\delta^2\sim (z_3 - z_2)(z_3 - z_1)\). All in all, we have
\begin{equation}
  \bm{\tau}_h(\bm{z}, \bm{v}) = \lim_{z_2, z_3 \to z_1}\frac{(z_2 - z_1)(z_3 - z_1)(z_3 - z_2)}{\Delta(\bm{z})}
  \det
  \begin{pmatrix}
    \frac{\mathcal{Y}_1(z_1)}{z_1 - v_1}  & \frac{\mathcal{Y}_1^{(1)}(z_1)}{z_1 - v_1} &
    \frac{\mathcal{Y}^{(2)}_1(z_1)}{z_1 - v_1} & \dots & \frac{\mathcal{Y}_1(z_n)}{z_n - v_1}\\
    \frac{\mathcal{Y}_2(z_1)}{z_1 - v_2} & \frac{\mathcal{Y}_2^{(1)}(z_1)}{z_1 - v_2} &
    \frac{\mathcal{Y}_2^{(2)}(z_1)}{z_1 - v_2} &  \dots & \frac{\mathcal{Y}_2(z_n)}{z_n - v_2}\\
    &  \vdots & & \\
    \frac{\mathcal{Y}_n(z_1)}{z_1 - v_n} & \frac{\mathcal{Y}_n^{(1)}(z_1)}{z_1 - v_n} &
    \frac{\mathcal{Y}_n^{(2)}(z_1)}{z_1 - v_n} &  \dots & \frac{\mathcal{Y}_n(z_n)}{z_n - v_n}
  \end{pmatrix}\; .
\end{equation}
Repeating the same idea for all columns, we get that the multiplicative factor cancels out, and
that the homogeneous limit gives 
\begin{equation}
  \bm{\tau}_h(z_1, \bm{v}) =
  \det
  \begin{pmatrix}
    \frac{\mathcal{Y}_1(z_1)}{z_1 - v_1}  & \frac{\mathcal{Y}_1^{(1)}(z_1)}{z_1 - v_1} &
    \frac{\mathcal{Y}^{(2)}_1(z_1)}{z_1 - v_1} & \dots & \frac{\mathcal{Y}_1^{(n-1)}(z_1)}{z_1 - v_1}\\
    \frac{\mathcal{Y}_2(z_1)}{z_1 - v_2} & \frac{\mathcal{Y}_2^{(1)}(z_1)}{z_1 - v_2} &
    \frac{\mathcal{Y}_2^{(2)}(z_1)}{z_1 - v_2} &  \dots & \frac{\mathcal{Y}_2^{(n-1)}(z_1)}{z_1 - v_2}\\
    &  \vdots & & \\
    \frac{\mathcal{Y}_n(z_1)}{z_1 - v_n} & \frac{\mathcal{Y}_n^{(1)}(z_1)}{z_1 - v_n} &
    \frac{\mathcal{Y}_n^{(2)}(z_1)}{z_1 - v_n} &  \dots & \frac{\mathcal{Y}^{(n-1)}_n(z_1)}{z_1 - v_n}
  \end{pmatrix}\; .
\end{equation}

Finally, let we write \(z_1 \equiv w\) and using elementary row operations we find
\begin{equation}
  \bm{\tau}_h(w, \bm{v}) =
  \frac{1}{\prod_{k=1}^n(w - v_k)}
  \mathcal{W}[\mathcal{Y}_1, \mathcal{Y}_2, \dots, \mathcal{Y}_n](w)
  \; .
\end{equation}
that agrees with the results we obtained above. 

\end{subequations}


\subsection{Problems}
Here are some important comments and problems to discuss. 

It seems that we can consider a basis of functions \(\mathcal{Y}_j(z)\) for the Baker-Akhiezer
functions. I think that this object might say something interesting. In particular, we can
explicitly build solutions of the KP equation. In fact, the functions \(\mathcal{Y}_j\)
above have poles at the Bethe roots, it is interesting. 
\begin{itemize}
\item[\faCheck] What is the pole and root structure of the tau and Baker-Akhiezer (BA) functions?
  In particular, what happens at the Bethe roots?
\item[\faCheck] Basis tau functions
\item[\faCheck] Write the cases n=2 and n=3
\item[\faCheck] What the general case for all \(z_k\) approaching to a single
  value \(w\). 
\item Write these tau functions in terms of Miwa coordinates
\item It turns out that the Baker-Akhiezer function is the appropriate object to discuss
  the Bethe roots. 
  \item I think that these tau functions come from algebro geometric tau functions. 
  \item I can try to study the dynamics closer to the Bethe roots, I think that the
    \(\Omega\)'s will have some interesting expansions, some of them might vanish. 
  \item This might become a Riemann Hilbert problem
  \item Are these tau functions associated to the Calogero-Moser model??
  \item Elliptic case - see slavnov, zabrodin's paper
\end{itemize}



\subsubsection*{Acknowledgments}
I would like to thank...

\printbibliography

\end{document}
